

This constitutes a simple and compact framework for calculating conduction speed and associated delays for individual axons, and has been used widely in computational and theoretical studies on a diverse range of subjects such as dynamics \citep{bojak2010axonal}, species differences in information processing capacity \citep{caminiti2009evolution}, and efficiency of brain organization \citep{chomiak2009what}. However, for the meso- and macro-scale level of neural populations that is typically of interest in neuroimaging, it is necessary to extend the single axon model of Eqs. 1-2 to  populations of axons, that may vary considerably in $d_m$ and $g$, and even $l$. 

Light and electron microscopy images of the cross-sections of nerve fascicles show that they comprise a mixture of myelinated and unmyelinated fibres of various sizes\footnote{We return to the interesting and challenging case of unmyelinated fibres later sections; for present purposes we focus on myelinated fibres only}. Critically, the profile of axon diameters, or \textit{axon diameter distribution} (ADD), varies across species \citep{caminiti2013diameter}, across brain regions within a species \citep{aboitiz1992fiber,innocenti2010fiber}, and is modified by ageing and neurodegenerative disease \citep{peters2009the}. Thus the ADD (together with other information such as estimates of axonal density) represents an anatomically concrete and physiologically meaningful marker of white matter status\footnote{Unlike, for example, popular diffusion MRI-based metrics such as fractional anisotropy (FA)}. 
