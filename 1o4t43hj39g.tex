

This constitutes a simple and compact framework for calculating conduction speed and associated delays for individual axons, and has been used widely in computational and theoretical studies on a diverse range of subjects such as dynamics \cite{bojak2010axonal}, species differences in information processing capacity \cite{caminiti2009evolution}, and efficiency of brain organization \cite{chomiak2009what}. However, for the meso- and macro-scale level of neural populations that is typically of interest in neuroimaging, it is necessary to extend the single axon model of Eqs. 1-2 to  populations of axons, that may vary considerably in $d_m$ and $g$, and even $l$. 

Light and electron microscopy images of the cross-sections of nerve fascicles show that they comprise a mixture of myelinated and unmyelinated fibres of various sizes\footnote{We return to the interesting and challenging case of unmyelinated fibres later sections; for present purposes we focus on myelinated fibres only}. Critically, the profile of axon diameters, or \textit{axon diameter distribution} (ADD), varies across species \cite{caminiti2013diameter}, across brain regions within a species \cite{aboitiz1992fiber,innocenti2010fiber}, and is modified by ageing and neurodegenerative disease \cite{peters2009the}. Thus the ADD (together with other information such as estimates of axonal density) represents an anatomically concrete and physiologically meaningful marker of white matter status\footnote{Unlike, for example, popular diffusion MRI-based metrics such as fractional anisotropy (FA)}. 

Fits to histograms of axon diameter counts indicate that myelinated fibre ADDs are well-described by a gamma distribution, with peak (mode) around 1$\mu$m and a long tail extending up to approximately 10$\mu$m. \cite{aboitiz1992fiber}. Several studies have used these microscopy-derived ADD histograms, together with the above equations, to obtain estimates the \textit{conduction velocity distribution} (CVD) and \textit{conduction delay distribution} (CDD) associated with a given white matter fascicle. This is done by computing Eq. 2 for all diameters within the ranges measured, and and assigning the CVD and CDD bin weightings directly from the corresponding ADD bin weightings. Whilst this approach of treating each histogram bin separately is the most empirically accurate and assumption-free way of computing CDDs (when an empirical ADD histogram is available), it is problematic when we want to know about uncertainty. This is due to the non-independence of nearby bins in the axon size count histograms. If, for example, we calculated variability (e.g. standard deviation over samples) in histogram bin weightings for different sections of the corpus callosum, and summed this uncertainty over bins, we would get a massively over-inflated estimate of the resultant total uncertainty, because adjacent bin weights will be highly correlated and therefore do not contribute independently to the overall uncertainty. 

For this reason it is preferable for present purposes to work directly with probability density functions summarizing ADD profiles, which have low degrees of freedom (generally 2 parameters), and therefore minimal redundancy. In the following we therefore represent the population of diameters within a fibre bundle by the gamma-distributed random variable $\mathbf{X_D} \sim G(x,k,\theta)$\footnote{We note that alternative and potentially superior parametric distributions have been proposed for the ADD (e.g. Sepehrband et al. \cite{sepehrband2016parametric}). Here we restrict ourselves to the gamma distribution since it has been the most widely used to date; however the majority of the following is easily adapted to alternative distribution functions.}, whose probability density function $G(x)$ for scale and shape parameters $\theta$ and $k$, evaluated at a given diameter $x$, is given by 
