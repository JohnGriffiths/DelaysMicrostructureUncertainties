
Something about delays in lefebvre models...


In principle, fixing conduction velocity from microstructure variables will allow this parameter to be fixed, thus allowing more flexibility in exploring and estimating other important parameters such as global gains and local excitatory-inhibitory interactions. Relatedly, some modeling approaches look to estimate conduction delays from neurophysiology data (Robinson model papers; DCM papers). This can be highly problematic (David refs). Microstructure data would be helpful here in constraining, and/or fixing, delay parameters. 
A second potentially important contribution of microstructure-estimated CDDs is that they have the potential to modify substantially the relative delays of network connections. In PSEs where conduction velocity is varied globally, the delays of each network connection may all increase and decrease, but their relative values (determined by tract lengths) remains constant. An important empirical question, for which there is currently very little relevant data in humans, is how much this relative delay structure, imposed essentially by inter-regional distances, is modified by per-tract differences in axonal calibers. 

A number of authors have studied neural mass and field models with distributed delays (e.g. Roberts and Robinson 2009; Bojak & Liley 2010; Hutt et al. 2006). 

%\citeyear{roberts 2009; Bojak \& Liley \citeyear{bojak2010axonal}; Hutt et al. 2006). Some sentences on this...


% the implicit assumption of using a single delay is that that is the most common delay or something like that
% ...which is what we're characterizing with the Expectation estimates

% 