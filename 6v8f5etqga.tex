

\begin{figure}[h!]
\begin{center}
\includegraphics[width=10cm]{Chapters_1_2_3_4_5_6_7_666_1.png}% This is a *.jpg file
\end{center}
\caption[\emph{Diameter distribution gamma parameters. }]{\emph{Diameter distribution gamma parameters. }Left panel: Gamma distributions fitted to the axon diameter histograms of Aboitiz et al. (1992), 
figure 4. The scale and shape parameters are listed after the corpus callosum region name 
in the figure legend. Right panel: regression of the scale and shape parameters in top panel. 
Red markers indicate the parameters from the five distributions shown in top panel;
the regression line and equation were fitted to these data. Blue markers indicate the 
three parameter pairs used in this study, sampled at three points evenly along the x axis, with the regression equation used to define the corresponding y values.  Uncertainty propagation analyses in this study chose a value for the diameter distribution  scale parameter in the mid-point of its range (0.35), with uncertainties of 0.2.}\label{fig:4}
\end{figure}



\subsubsection*{\textit{G-ratio}}

Whilst g-ratios have been reported as low as 0.2 and as high as 0.8, the
remarkable consistency of this quantity around 0.6-0.7 has been the
subject of several theoretical research papers (Rushton, 1951; Chomiak
\& Hu 2009; Paus \& Toro, 2009; Caminiti et al., 2009). This parameter
was therefore assigned a modest level of uncertainty: $g$ = 0.7 $_{+/-}$
0.1. 

\subsubsection*{\textit{Proportion of myelinated vs. unmyelinated fibres}}

Uncertainty in the proportion of myelinated fibres was
also set to a modest level of $p_{m}$ = 0.8 $_{+/-}$ 0.1, which covers
the majority of conventionally cited proportions of myelinated to
unmyelinated in non-diseased brains  \shortcite{bowley2010age}.

 
\subsection{Software note}

All code and data used in this study is freely available at \url{https://github.com/JohnGriffiths/DelaysMicrostructureUncertaintyEvaluation}

