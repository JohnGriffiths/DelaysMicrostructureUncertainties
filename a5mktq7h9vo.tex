
\subsection{Estimating conduction delays from microscopy and microstructure imaging data}

% - the key thing for this paper is that there is a diameter distribution and associated velocity and delay distribution 

Broadly speaking, conduction delays may be estimated from empirical data in one of four ways, 
  
\begin{enumerate}
\item Direct (neural or behavioural) response latency measurements such as ERPs, reaction times, or motor nerve conduction times  \cite{reed1991arm,allison1984developmental, marzi1999the,lodhia2017decreased,wishart1995interhemispheric}
\item Identification of lags, cross-correlations, MVAR model orders; or phase lags \cite{cimponeriu2004estimation,brovelli2004beta}

\item Parameter estimation from multivariate time- or frequency-domain biophysical network models \cite{friston2011dcm, kerr2011modelbased}.

\item Calculation from relevant microstructural variables (axon diameter, myelin thickness, fibre length) \cite{caminiti2009evolution,caminiti2013diameter,ringo1994time}
\end{enumerate}

Note that many techniques may fall into one or several of these categories, and will differ in multiple additional ways such as the type of delay being measured (e.g. individual cells; population average; population fastest; etc.) and how the delay is calculated and parametrized in a given model. Categories 1-3 above all operate on neurophysiological or behavioural data and rely on models of varying complexity that utilize timing relationships between measured variables, or relative to a stimulus or motor response, to identify delays. In contrast, category 4 - which shall be our focus in this paper - does not directly use any measurements of neural activity. Rather, it exploits known properties relating delays to the microstructural anatomy of axons, as well as the statistics of the axonal populations that comprise white matter fascicles in the central and peripheral nervous systems. In the following we discuss the details and assumptions of this approach, first in the context of individual axons, and then our extension to populations of axons with a distribution of axonal calibers. 


It was noted as early as the 1930s that the conduction velocity $v_m$ of myelinated axons is approximately linearly related to its outer fibre diameter $F_m$ (the diameter of the outermost part of the myelin sheath), with a coefficient $c_m$ of ~0.7  \shortcite{hursh1939the,rushton1951a,waxman1972relative}. The exact value of this coefficient has been debated and revised, but the general idea has been pretty consistent for some time. Since the $F$ is a composite of two quantities (axonal diameter and myelin thickness), it is useful to decompose it into $F_m = g/d_m$. Whilst the 'g-ratio' $g$ has been reported as low as 0.2 and as high as 0.8, the remarkable consistency of this quantity around 0.6-0.7 has been the subject of several theoretical research papers 
\shortcite{rushton1951a,chomiak2009what,paus2009could,caminiti2009evolution,ritchie1982on}. Thus $v_m$ and the individual myelinated fibre conduction \textit{delay} $\delta_m$ can be computed from $d_m$, $g$, and the axonal length $l$ as follows

\begin{eqnarray}
v_m      =& g^{-1} d_m c_m \\
\delta_m =& l^{-1} v_m
\end{eqnarray}




This constitutes a simple and compact framework for calculating conduction speed and associated delays for individual axons, and has been used widely in computational and theoretical studies on a diverse range of subjects such as dynamics \shortcite{bojak2010axonal}, species differences in information processing capacity \shortcite{caminiti2009evolution}, and efficiency of brain organization \shortcite{chomiak2009what}. However, for the meso- and macro-scale level of neural populations that is typically of interest in neuroimaging, it is necessary to extend the single axon model of Eqs. 1-2 to  populations of axons, that may vary considerably in $d_m$ and $g$, and even $l$. 

Light and electron microscopy images of the cross-sections of nerve fascicles show that they comprise a mixture of myelinated and unmyelinated fibres of various sizes\footnote{We return to the interesting and challenging case of unmyelinated fibres later sections; for present purposes we focus on myelinated fibres only}. Critically, the profile of axon diameters, or \textit{axon diameter distribution} (ADD), varies across species \shortcite{caminiti2013diameter}, across brain regions within a species \shortcite{aboitiz1992fiber,innocenti2010fiber}, and is modified by ageing and neurodegenerative disease \shortcite{peters2009the}. Thus the ADD (together with other information such as estimates of axonal density) represents an anatomically concrete and physiologically meaningful marker of white matter status\footnote{Unlike, for example, popular diffusion MRI-based metrics such as fractional anisotropy (FA)}. 

Fits to histograms of axon diameter counts indicate that myelinated fibre ADDs are well-described by a gamma distribution, with peak (mode) around 1$\mu$m and a long tail extending up to approximately 10$\mu$m. \shortcite{aboitiz1992fiber}. Several studies have used these microscopy-derived ADD histograms, together with the above equations, to obtain estimates the \textit{conduction velocity distribution} (CVD) and \textit{conduction delay distribution} (CDD) associated with a given white matter fascicle. This is done by computing Eq. 2 for all diameters within the ranges measured, and and assigning the CVD and CDD bin weightings directly from the corresponding ADD bin weightings. Whilst this approach of treating each histogram bin separately is the most empirically accurate and assumption-free way of computing CDDs (when an empirical ADD histogram is available), it is problematic when we want to know about uncertainty. This is due to the non-independence of nearby bins in the axon size count histograms. If, for example, we calculated variability (e.g. standard deviation over samples) in histogram bin weightings for different sections of the corpus callosum, and summed this uncertainty over bins, we would get a massively over-inflated estimate of the resultant total uncertainty, because adjacent bin weights will be highly correlated and therefore do not contribute independently to the overall uncertainty. 

For this reason it is preferable for present purposes to work directly with probability density functions summarizing ADD profiles, which have low degrees of freedom (generally 2 parameters), and therefore minimal redundancy. In the following we therefore represent the population of diameters within a fibre bundle by the gamma-distributed random variable $\mathbf{X_D} \sim G(x,k,\theta)$\footnote{We note that alternative and potentially superior parametric distributions have been proposed for the ADD (e.g. Sepehrband et al. \citeyearNP{sepehrband2016parametric}). Here we restrict ourselves to the gamma distribution since it has been the most widely used to date; however the majority of the following is easily adapted to alternative distribution functions. 
}, whose probability density function $G(x)$ for scale and shape parameters $\theta$ and $k$, evaluated at a given diameter $x$, is given by 


\begin{eqnarray}
G(x,k,\theta) =& \frac{\displaystyle x^{k^{-1} exp(- \frac{\displaystyle x}{ \displaystyle \theta})}}{\displaystyle \theta^{k} \Gamma(k)} 
\end{eqnarray}

where $\Gamma$ is the gamma function. 

Using this representation, we can now compute the CVDs and CDDs directly for the entire distribution simultaneously, rather than from individual bins. This is essential for our application to uncertainty estimation, but may also prove useful in other contexts.


We first note the following two properties of gamma-distributed random variables: 

\begin{eqnarray}
a \mathbf{X} =& G(k,a \theta ) \\
\mathbf{X}^{-1} =& G^{-1}(k,\theta^{-1} )
\end{eqnarray}

where $G^{-1}$ is an inverse gamma distribution

\begin{eqnarray}
G^{-1}(x,k,\theta) =& \frac{\displaystyle x^{-k-1} exp(- \frac{\displaystyle \theta}{ \displaystyle k})}{\displaystyle \theta^{k} \Gamma(k)} 
\end{eqnarray}

From this is follows that the CVD $\mathbf{V_m}$ and CDD $\mathbf{\Delta_m}$ for a myelinated fibre bundle of length $L$, g-ratio $g$, and ADD scale and shape parameters $k_m$ and $\theta_{D_{m}}$, are given by 

\begin{eqnarray}
\mathbf{V_m} =& G(k_m, \theta_{V_{m}}) \\
\mathbf{\Delta_m} =& G^{-1}(k_m, \theta_{\Delta_{m}})
\end{eqnarray}

where 

\begin{eqnarray}
\theta_{V{_m}} =& g^{-1} \theta_{D_{m}}  \\
\theta_{\Delta{_m}} =& L^{-1} \theta_{V_{m}}
\end{eqnarray}

Finally, the mean $\mu_{\Delta_m}$ (or expectation $E_{\Delta_m}$) of $\mathbf{\Delta}_m$ is given arithmetically by 

\begin{eqnarray}
\mu_{\Delta_m} =& \frac{\displaystyle k_m}{\displaystyle \theta_{\Delta_m} -1} \end{eqnarray}

and the mode $M_{\Delta_m}$ by 

\begin{eqnarray}
M_{\Delta_m} =& \frac{\displaystyle k_m}{\displaystyle \theta_{\Delta_m} +1}
\end{eqnarray}

A summary of these calculations is shown in figure 1. 





\begin{figure}[h!]
\begin{center}
%\includegraphics[width=10cm]{Chapters_1_2_3_4_5_6_7_666_1.png}% This is a *.jpg file
\end{center}
\caption[\emph{From white matter microstructure to conduction delay distributions}]{\emph{From white matter microstructure to conduction delay distributions}. Figure here with diagram of fibre bundle cross section showing distribution of diameters and myelin thickness; also equations for calculations and transforms from ADD->CVD->CDD} \label{fig:1}
\end{figure}

%The chief motivation of the present study is the techniques developed by Assaf, Alexander, and others. In particular, AxCaliber allows estimation the ADD gamma distribution parameters. 

%In the next sections we detail our methodological approach to computing CDD uncertainties, and sensitivities for the various parameters in the above equations. 

\subsection{Present Study}

The principal aim of the present study is to examine the how estimates of various quantities relating to$\mathbf{\Delta}_m$ - chiefly $\theta_{\Delta_m}$ and $M_{\Delta_m}$, depend on uncertainty in the values of $L$,$k$, and $\theta_m$, and $g$, given the both the typical ranges of these parameters and the estimates that can potentially be obtained from noninvasive imaging. Our focus here is loosely on MR imaging in humans, but the general conclusions and techniques used may also be applied in microscopy studies (e.g. Caminiti et al. \citeyearNP{caminiti2009evolution}, Innocenti et al. \citeyearNP{innocenti2010fiber}) in both human and non-human samples. In the following sections we now describe in detail the uncertainty evaluation procedure, as well as the ranges and expected values chosen for the key parameters. 


% - lengths. a lot of work just using lengths and assuming a value for velocity. 
% - now we can estimate diameters and g-ratios. 
% - there are limitations in these measurements but in principle they can be used to make calculations of the kind done above. this is what we're concerned with here.
 
%Conduction delays play a key role in sculpting patterns of brain dynamics at the macroscopic scale (Deco et al., 2009). Until relatively recently it has been necessary for researchers looking to model these phenomena to use heuristics such as head size or population averages from post-mortem studies (Nunez \& Srinivasen 2006) to estimate the length of fibre pathways in individual subjects. Nowadays, diffusion-weighted MR tractography provides a more direct and noninvasive measure of this anatomical feature, and a number of researchers have incorporated tractography-based connection length values into their models (Izhikevich \& Edelman, 2008; Valdes-Sosa et al., 2009). Many still use non-tractography proxy methods, however, based on e.g. Euclidean distance (e.g.~Ghosh et al., 2008; Deco et al., 2009) or volumetric segmentation of T1-weighted MR images (Bojak et al., 2011). As a large proportion of this research is concerned with developing either forward (e.g. Bojak et al., 2011) or inverse (e.g. David et al. 2006; Valdes-Sosa et al. 2009) models that capture subject-specific anatomy and neural activity patterns, the relative accuracy of these various non-tractography or non-subject-specific proxy measures of fibre length is an important practical question. Similarly, a question that is certain to receive growing attention in the coming decade is how the `new breed' of MRI-based tissue microstructure metrics might contribute to this enterprise. Recent developments in structural MR imaging are making measurements of both axon diameters (Assaf et al., 2006, 2007, 2008; Alexander et al., 2010), myelin thickness (Stikov et al., 2011), myelin dystrophy (Avram et al., 2010). Whether these developments play a peripheral or a central role in future models of human brain dynamics will be a function of the precision of the measurements and the degree of precision required in the modelling process.



%\subsection{Conduction delays and neuroanatomy}

%Conduction delays play a key role in sculpting patterns of brain dynamics at the macroscopic scale (Deco et al., 2009). Until relatively recently it has been necessary for researchers looking to model these phenomena to use heuristics such as head size or population averages from post-mortem studies (Nunez \& Srinivasen 2006) to estimate the length of fibre pathways in individual subjects. Nowadays, diffusion-weighted MR tractography provides a more direct and noninvasive measure of this anatomical feature, and a number of researchers have incorporated tractography-based connection length values into their models (Izhikevich \& Edelman, 2008; Valdes-Sosa et al., 2009). Many still use non-tractography proxy methods, however, based on e.g. Euclidean distance (e.g.~Ghosh et al., 2008; Deco et al., 2009) or volumetric segmentation of T1-weighted MR images (Bojak et al., 2011). As a large proportion of this research is concerned with developing either forward (e.g. Bojak et al., 2011) or inverse (e.g. David et al. 2006; Valdes-Sosa et al. 2009) models that capture subject-specific anatomy and neural activity patterns, the relative accuracy of these various non-tractography or non-subject-specific proxy measures of fibre length is an important practical question. Similarly, a question that is certain to receive growing attention in the coming decade is how the `new breed' of MRI-based tissue microstructure metrics might contribute to this enterprise. Recent developments in structural MR imaging are making measurements of both axon diameters (Assaf et al., 2006, 2007, 2008; Alexander et al., 2010), myelin thickness (Stikov et al., 2011), myelin dystrophy (Avram et al., 2010). Whether these developments play a peripheral or a central role in future models of human brain dynamics will be a function of the precision of the measurements and the degree of precision required in the modelling process.


%$\subsection{Present studies}


%The aim of the present study was to evaluate systematically the accuracy and potential sources of variability involved in both current uses of tractography and potential future uses of microstructure imaging for anatomically-informed models of neural dynamics.

% We used linear uncertainty propagation techniques to assess the extent to which accurate knowledge of fibre length, diameter distribution, and myelin thickness contribute to accurate estimates of CDDs.  For this I drew on the results from a number of histological and neuroimaging sources. 

% Tract length measures are important for models of spatially distant interacting neuronal populations that incorporate axonal conduction delays. The reason for assessing the accuracy of the various fibre length approximations described in the previous section is that this directly affects the precision of conduction delay estimates based on these measures. However, track length is not the sole source of uncertainty in these delay estimates: myelin thickness, proportion of myelinated vs.~unmyelinated fibres, and the distributions of myelinated and unmyelinated diameters are all known with varying degrees of uncertainty, and varying levels of possibly for improved measurement accuracy. I

%n order to properly quantify the contribution of uncertainties in each of these quantities to the total uncertainty in
%conduction delay estimates, I conducted a series of formal uncertainty
%evaluations, in accordance with the rules and guidelines set out by the
%Joint Committee for Guides in Metrology (2008).


\section{Methods}

\subsection{Uncertainty Evaluation}


In order to properly quantify the contribution of various sources of uncertainty in the CDD model outlined above, we conducted a series of formal uncertainty evaluations, in accordance with the rules and guidelines set out by the Joint Committee for Guides in Metrology (2008). An uncertainty evaluation begins by defining a function $f$, the \textit{measurement model}, that relates the \textit{measurand} (quantity of interest or outcome variable) $Y$ to the input quantities $X_{i}$; i.e. $Y = f(x_{1},...x_{n})$. Note that in our case the measurement model does not refer to the MRI microstructure or microscopy measurements process, but rather the computation of the CDD and derivative quantities given estimates of the various parameters discussed in the previous section. Once the measurement model is specified,
an uncertainty evaluation proceeds by obtaining expectations of the
probability distributions on the input quantities $X=X_{1}...,X_{N}$,
and the \emph{standard uncertainty}, or error, in those estimates. Expectations of the measurand $Y$ are then calculated straightforwardly from the measurement model and the expectations of $X$. The critical
part of the evaluation procedure is calculation of the standard
uncertainty of $Y$, which uses the mathematical technique of \emph{uncertainty} or \emph{error propagation}. For linear models this essentially consists of a weighted combination of the uncertainties of $X$, correcting for non-independence between inputs using covariance terms. Depending on the complexity of the model, this can be done using
analytically exact methods, analytic approximations, or sampling techniques such as Markov Chain Monte Carlo. The measurement models specified here were simple enough for an exact approach, which was implemented with custom error propagation routines modified from Rocklin et al. (\citeyearNP{rocklin2017symbolic}), using the python symbolic computing library \textit{sympy} \cite{meurer2017sympy}. The uncertainty evaluation procedure concludes by summarizing the results of uncertainty propagation. In addition to the standard uncertainty in $Y$, the model is summarized by the \emph{fractional} or \emph{relative uncertainty} $E/S$ (measurement of uncertainty divided by measured value) and \emph{expanded uncertainty} $FU=S/C$, which is the standard uncertainty multiplied by a coverage factor (an interval over which the majority of potential values of the measurand reside). Finally, \emph{sensitivity coefficients} $s_{1}...,s_{N}$ can be derived for each variable $X_{i}$ in the measurement model as the first order partial derivatives of $f$ with respect to $X_{i}$. These quantify the amount of influence that uncertainty in that variable has on the final estimate of the measurand.



\subsection{Measurement Models}

\subsubsection*{\textit{Model 1: Myelinated fibre delay distribution parameters}}

For the first set of uncertainty evaluations we define the measurand to be the myelinated fibres CDD scale parameter $\theta_{\Delta_m}$. Following eqs. 8-10, the measurement model is therefore given by $f_{1} = \theta_{\Delta_m} = L^{-1} g^{-1} \theta_{D_m}$. This provides a
`holistic' quantification of uncertainty in the entire CDD, since together with the shape parameter $k_{m}$ (which, conditional on $\theta_{m}$, has no additional uncertainty; see below) it defines the probability weighting of any delay lying within the support of $\Delta_{m}$. 


\subsubsection*{\textit{Model 2: Expected value of myelinated fibre delay distribution}}

Whilst $f_{1}$ provides an elegant and simple measurement model, it suffers from the problems that a) it is in units that have no direct physical or physiological interpretation, and b) the uncertainties propagated up from the model parameters are spread over the entire distribution in a somewhat unintuitive fashion. We therefore chose as the second measurement model the expected value or mean of the myelinated fibre CDD, as given by eq. 11: $f_2 = E_{\Delta_m} = \frac{k}{1-  k^{-1} (L^{-1} g^{-1} \theta_{D_m} - 1)}$. This is perhaps the most directly useful delay variable considered in this paper, as the overwhelming majority of modelling studies examining the effects of conduction delays use only a single delay for the connection between a given pair of brain regions, rather than multiple delays or full delay distributions. The use of a single delay can be understood as an approximation of the full delay distribution by its average (mean) or most common value (mode). We return to the issue of single vs. multiple/distributed delays in the discussion. 

%Range of values over which to vary delays over which microstructure-constrained delays may be varied?


\subsubsection*{\textit{Model 3: Expected value of mixed myelinated + unmyelinated delay distribution}}

%The first and second measurement models are based directly on eqs. 8-10, and use as measurands  $\theta_{{D_m}}$ and $\mu_{D_{m}}$, respectively. 

%Model 1 therefore quantifies, in a holistic fashion, uncertainty in the overall shape of the CDD. Model 2, in contrast, focuses on the expectation or central tendency of the CDD. This can be understood as giving a single, representative number for the conduction delay of a given fibre pathway, which is commonly done for the sake of simplification in modelling studies, 


%The first measurement model defines the measurand to be the myelinated fibres delay distribution ($\Delta_{m}$) scale parameter $\theta_{m} '$, making the measurement model $f_{1} = \theta_{m} ' = \frac{Lg\theta_{m}}{c_{m}}$. 


%I consider
%here two sets of measurement models based on the CDD concept discussed above. 

%In that chapter the myelinated and unmyelinated fibre delay distributions $\Delta_{m}$ and $\Delta_{u}$ were computed by sampling from gamma distributions of axon diameters and applying a simple transform to compute the delay for a given axon given its diameter. These were then pooled with relative proportions of samples $p_{m}$ and $p_{u} = 1-p_{m}$ to give a sampling-based mixture distribution $\Delta_{m+u}$. Here I use an alternative parameterization of the delay distributions, which uses an analytic rather than a sampling-based procedure for the computation of $\Delta_{m}$ and $\Delta_{u}$, and so is more amenable to analysis using linear error propagation techniques. The key difference is that the expressions for conductions for delays as a function of diameter $d$ are applied directly to the scale parameter of the fibre diameter distributions. The result is a mixture of \emph{inverse} gamma
%distributions; $\Delta_{m+u} = p_{m} \Gamma^{-1}(k_{m},\frac{Lg\theta_{m}}{c_{m}}) + p_{u} \Gamma^{-1}(k_{u}\frac{L}{\theta_{u}c_{u}})$.

Unmyelinated fibres are largely ignored in the microscopy and emerging neuroimaging literature on ADDs, primarily for the understandable reason that they are extremely difficult to measure from accurately due to their small size (mostly <1$\mu$m), which makes them effectively invisible to both light microscopy and MRI microstructure imaging.  Because electron microscopy is impossible in post-mortem human tissue due to the need for toxic tracer injections before death, estimates of unmyelinated fibre densities and diameters need to be extrapolated from studies in non-human primates such as the macaque. Importantly, such studies have observed as high as 40\% of axons to be unmyelinated in some brain locations \shortcite{bowley2010age}. A 'complete' CDD would therefore include contributions from both myelinated and unmyelinated fibres, and it is immediately evident that this could have a substantially longer tail than the CDD for myelinated fibres alone, given the substantially slower conduction velocities of unmyelinated fibers. We therefore included as our third measurement model the expectation of the mixed myelinated+unmyelinated fibre delay distribution; $f_{3} = E_{\Delta_{m+u}} = E(p_m  \mathbf{\Delta}_{m} + (1-p_m) \mathbf{\Delta}_{u}) $, where $p_m$ is the proportion of fibres in the fascicle that are unmyelinated. 

\textit{++TO DO: define $\Delta_u$}


% NEED TO DEFINE E_D_M+U...


%The procedure detailed above provides a general framework for assessing how uncertainty in length, g-ratio, and axon diameter distribution parameters impact on the precision on conduction delay estimates. As is normally the case with scientific measurements, these uncertainties have multiple origins. In the present case these can be separated first and foremost into within-subject variability, between-subject variability, and measurement error. The importance of the former depends on the magnitude of the latter: measurements are accurate only insofar as they increase the confidence in the estimate of a quantity beyond that of an estimate based only on the population variation. For the case of length, both the tractography and the T1 proxy measurements have relatively high accuracy, and so we can focus on the difference in precision between the two. For $g$ and $\theta_{m}$, the accuracy of the measurement techniques is lower, and so we are concerned more with the comparison of measurement vs.~no measurement at all. Parallelling this, the pragmatic reality is that neuroimaging studies at present collect only T1-weighted images as standard. Diffusion-weighed images for tractography are reasonably common, but the multiple b-value diffusion-weighted sequences necessary for axon diameter imaging, and the magnetization transfer sequences necessary for g-ratio imaging, remain highly specialist technologies. Thus the question for noninvasive estimates of length is one of `which', whilst for g-ratio and axon diameters is still one of `whether'. The uncertainty values compared in this study were chosen to reflect this situation.


\subsection{Parameter values}

The three measurement models $f_{1}$, $f_{2}$, and $f_{3}$ share input parameters
$L$, $\theta_{m}$, $k_{m}$, $g$, and $c_{m}$, with $f_{3}$ additionally
having parameters $\theta_{u}$, $k_{u}$, and $p_{m}$. None of these may
be said to be known perfectly and entirely without uncertainty. We restrict our focus in this study to uncertainty in parameters whose precision may be actually or potentially be improvable in the near future using noninvasive imaging: $L$, $g$, $\theta_{m}$, $p_{m}$. Each of these are considered with and without measurement uncertainty, and error propagation calculations were made for all permutations of zero and non-zero uncertainties in these parameters. The remaining parameters were set as constants. For all parameters apart from tract length $L$ (see below), zero uncertainty level should be understood as an idealization that is useful in understanding their respective contributions to the combined standard uncertainty. Non-zero levels were chosen to reflect the likely range of values which the parameter may take, considering both inter-regional and inter-subject variation.


\subsubsection*{\textit{Tract Length}}

The lengths of white matter fibre tracts vary widely in humans, from around 10mm to as high as 200mm in large brains. However, in contrast to the other model parameters specifying ADDs and myelination, it is not necessary to incorporate this full range of variation into the measurement model, because in practice additional information is available that substantially reduces uncertainty about tract length. In particular, diffusion MRI tractography allows fairly unambiguous measurement of tract lengths to a high degree of accuracy. Tract lengths can be also approximated to fairly high precision (10mm error) without use of tractography, simply using using Euclidean distance. It should be noted however that, depending on how the brain regions for which delay is being estimated are defined, there may be a genuine distribution of tract lengths associated with a given anatomical connection. Naturally, the width of this distribution will be larger when larger regions are used. We therefore set $L$ = 160mm+/- 20mm.

%uncertainty in length to X, for reason X (Euclidean distance approximation / length dispersion).


\subsubsection*{\textit{Axon diameters}}

Uncertainties for $\theta_{m}$ were based on the data reported in Aboitiz et al. (\citeyearNP{aboitiz1992fiber}). These authors used light microscopy to examine the five main segments of the corpus callosum (genu, anterior body, mid body, posterior body, and splenium). They presented the diameter distributions of each segment as histograms (figure 4 of that paper). To identify the physiological range of $k_{m}$ and $\theta_{m}$, the parameters of five gamma distributions were fitted to the histogram data presented in that paper (figure 4.4, left panel). This yielded a range of 2-5 for $\theta_{m}$, and so in this study $\theta_{m}$ was set to lie in the middle of this range, with an uncertainty that spans the majority of this range; 3.5+/-2. Because $\theta_{m}$ and $k_{m}$ are highly correlated, the appropriate value of $k_{m}$ was selected by regressing $k_{m}$ onto the $\theta_{m}$ (figure 3, right panel), and
using the regression model to specify the corresponding $k_{m}$ for a given $\theta_{m}$. Thus, for example, $\theta_m$=3.5 specified $k_m$=4.12.



\begin{figure}[h!]
\begin{center}
\includegraphics[width=10cm]{Chapters_1_2_3_4_5_6_7_666_1.png}% This is a *.jpg file
\end{center}
\caption[\emph{Diameter distribution gamma parameters. }]{\emph{Diameter distribution gamma parameters. }Left panel: Gamma distributions fitted to the axon diameter histograms of Aboitiz et al. (1992), 
figure 4. The scale and shape parameters are listed after the corpus callosum region name 
in the figure legend. Right panel: regression of the scale and shape parameters in top panel. 
Red markers indicate the parameters from the five distributions shown in top panel;
the regression line and equation were fitted to these data. Blue markers indicate the 
three parameter pairs used in this study, sampled at three points evenly along the x axis, with the regression equation used to define the corresponding y values.  Uncertainty propagation analyses in this study chose a value for the diameter distribution  scale parameter in the mid-point of its range (0.35), with uncertainties of 0.2.}\label{fig:4}
\end{figure}



\subsubsection*{\textit{G-ratio}}

Whilst g-ratios have been reported as low as 0.2 and as high as 0.8, the
remarkable consistency of this quantity around 0.6-0.7 has been the
subject of several theoretical research papers (Rushton, 1951; Chomiak
\& Hu 2009; Paus \& Toro, 2009; Caminiti et al., 2009). This parameter
was therefore assigned a modest level of uncertainty: $g$ = 0.7 $_{+/-}$
0.1. 

\subsubsection*{\textit{Proportion of myelinated vs. unmyelinated fibres}}

Uncertainty in the proportion of myelinated fibres was
also set to a modest level of $p_{m}$ = 0.8 $_{+/-}$ 0.1, which covers
the majority of conventionally cited proportions of myelinated to
unmyelinated in non-diseased brains  \shortcite{bowley2010age}.

 
\subsection{Software note}

All code and data used in this study is freely available at \url{https://github.com/JohnGriffiths/DelaysMicrostructureUncertaintyEvaluation}