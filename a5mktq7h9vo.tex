
\subsection{Estimating conduction delays from microscopy and microstructure imaging data}

% - the key thing for this paper is that there is a diameter distribution and associated velocity and delay distribution 

Broadly speaking, conduction delays may be estimated from empirical data in one of four ways, 
  
\begin{enumerate}
\item Direct (neural or behavioural) response latency measurements such as ERPs, reaction times, or motor nerve conduction times  \cite{reed1991arm,allison1984developmental, marzi1999the,lodhia2017decreased,wishart1995interhemispheric}
\item Identification of lags, cross-correlations, MVAR model orders; or phase lags \cite{cimponeriu2004estimation,brovelli2004beta}

\item Parameter estimation from multivariate time- or frequency-domain biophysical network models \cite{friston2011dcm, kerr2011modelbased}.

\item Calculation from relevant microstructural variables (axon diameter, myelin thickness, fibre length) \cite{caminiti2009evolution,caminiti2013diameter,ringo1994time}
\end{enumerate}

Note that many techniques may fall into one or several of these categories, and will differ in multiple additional ways such as the type of delay being measured (e.g. individual cells; population average; population fastest; etc.) and how the delay is calculated and parametrized in a given model. Categories 1-3 above all operate on neurophysiological or behavioural data and rely on models of varying complexity that utilize timing relationships between measured variables, or relative to a stimulus or motor response, to identify delays. In contrast, category 4 - which shall be our focus in this paper - does not directly use any measurements of neural activity. Rather, it exploits known properties relating delays to the microstructural anatomy of axons, as well as the statistics of the axonal populations that comprise white matter fascicles in the central and peripheral nervous systems. In the following we discuss the details and assumptions of this approach, first in the context of individual axons, and then our extension to populations of axons with a distribution of axonal calibers. 


It was noted as early as the 1930s that the conduction velocity $v_m$ of myelinated axons is approximately linearly related to its outer fibre diameter $F_m$ (the diameter of the outermost part of the myelin sheath), with a coefficient $c_m$ of approximately 0.7  \citep{hursh1939the,rushton1951a,waxman1972relative}. The exact value of this coefficient has been debated and revised, but estimates have not changed substantially in the past 50 years. Since the $F$ is a composite of two quantities (axonal diameter and myelin thickness), it is useful to decompose it into $F_m = g/d_m$. Whilst the 'g-ratio' $g$ has been reported as low as 0.2 and as high as 0.8, the remarkable consistency of this quantity around 0.6-0.7 has been the subject of several theoretical research papers 
\cite{rushton1951a,chomiak2009what,paus2009could,caminiti2009evolution,ritchie1982on}. Thus $v_m$ and the individual myelinated fibre conduction \textit{delay} $\delta_m$ can be computed from $d_m$, $g$, and the axonal length $l$ as follows

\begin{eqnarray}
v_m      =& g^{-1} d_m c_m \\
\delta_m =& l^{-1} v_m
\end{eqnarray}
