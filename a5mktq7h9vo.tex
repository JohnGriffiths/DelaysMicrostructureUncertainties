
\subsection{Estimating conduction delays from microscopy and microstructure imaging data}

% - the key thing for this paper is that there is a diameter distribution and associated velocity and delay distribution 

Broadly speaking, conduction delays may be estimated from empirical data in one of four ways, 
  
\begin{enumerate}
\item Direct (neural or behavioural) response latency measurements such as ERPs, reaction times, or motor nerve conduction times  \cite{reed1991arm,allison1984developmental, marzi1999the,lodhia2017decreased,wishart1995interhemispheric}
\item Identification of lags, cross-correlations, MVAR model orders; or phase lags \cite{cimponeriu2004estimation,brovelli2004beta}

\item Parameter estimation from multivariate time- or frequency-domain biophysical network models \cite{friston2011dcm, kerr2011modelbased}.

\item Calculation from relevant microstructural variables (axon diameter, myelin thickness, fibre length) \cite{caminiti2009evolution,caminiti2013diameter,ringo1994time}
\end{enumerate}

Note that many techniques may fall into one or several of these categories, and will differ in multiple additional ways such as the type of delay being measured (e.g. individual cells; population average; population fastest; etc.) and how the delay is calculated and parametrized in a given model. Categories 1-3 above all operate on neurophysiological or behavioural data and rely on models of varying complexity that utilize timing relationships between measured variables, or relative to a stimulus or motor response, to identify delays. In contrast, category 4 - which shall be our focus in this paper - does not directly use any measurements of neural activity. Rather, it exploits known properties relating delays to the microstructural anatomy of axons, as well as the statistics of the axonal populations that comprise white matter fascicles in the central and peripheral nervous systems. In the following we discuss the details and assumptions of this approach, first in the context of individual axons, and then our extension to populations of axons with a distribution of axonal calibers. 


It was noted as early as the 1930s that the conduction velocity $v_m$ of myelinated axons is approximately linearly related to its outer fibre diameter $F_m$ (the diameter of the outermost part of the myelin sheath), with a coefficient $c_m$ of ~0.7  \cite{hursh1939the,rushton1951a,waxman1972relative}. The exact value of this coefficient has been debated and revised, but the general idea has been pretty consistent for some time. Since the $F$ is a composite of two quantities (axonal diameter and myelin thickness), it is useful to decompose it into $F_m = g/d_m$. Whilst the 'g-ratio' $g$ has been reported as low as 0.2 and as high as 0.8, the remarkable consistency of this quantity around 0.6-0.7 has been the subject of several theoretical research papers 
\cite{rushton1951a,chomiak2009what,paus2009could,caminiti2009evolution,ritchie1982on}. Thus $v_m$ and the individual myelinated fibre conduction \textit{delay} $\delta_m$ can be computed from $d_m$, $g$, and the axonal length $l$ as follows

\begin{eqnarray}
v_m      =& g^{-1} d_m c_m \\
\delta_m =& l^{-1} v_m
\end{eqnarray}

 counts indicate that myelinated fibre ADDs are well-described by a gamma distribution, with peak (mode) around 1$\mu$m and a long tail extending up to approximately 10$\mu$m. \cite{aboitiz1992fiber}. Several studies have used these microscopy-derived ADD histograms, together with the above equations, to obtain estimates the \textit{conduction velocity distribution} (CVD) and \textit{conduction delay distribution} (CDD) associated with a given white matter fascicle. This is done by computing Eq. 2 for all diameters within the ranges measured, and and assigning the CVD and CDD bin weightings directly from the corresponding ADD bin weightings. Whilst this approach of treating each histogram bin separately is the most empirically accurate and assumption-free way of computing CDDs (when an empirical ADD histogram is available), it is problematic when we want to know about uncertainty. This is due to the non-independence of nearby bins in the axon size count histograms. If, for example, we calculated variability (e.g. standard deviation over samples) in histogram bin weightings for different sections of the corpus callosum, and summed this uncertainty over bins, we would get a massively over-inflated estimate of the resultant total uncertainty, because adjacent bin weights will be highly correlated and therefore do not contribute independently to the overall uncertainty. 

For this reason it is preferable for present purposes to work directly with probability density functions summarizing ADD profiles, which have low degrees of freedom (generally 2 parameters), and therefore minimal redundancy. In the following we therefore represent the population of diameters within a fibre bundle by the gamma-distributed random variable $\mathbf{X_D} \sim G(x,k,\theta)$\footnote{We note that alternative and potentially superior parametric distributions have been proposed for the ADD (e.g. Sepehrband et al. \cite{sepehrband2016parametric}). Here we restrict ourselves to the gamma distribution since it has been the most widely used to date; however the majority of the following is easily adapted to alternative distribution functions. 
}, whose probability density function $G(x)$ for scale and shape parameters $\theta$ and $k$, evaluated at a given diameter $x$, is given by 


\begin{eqnarray}
G(x,k,\theta) =& \frac{\displaystyle x^{k^{-1} exp(- \frac{\displaystyle x}{ \displaystyle \theta})}}{\displaystyle \theta^{k} \Gamma(k)} 
\end{eqnarray}

where $\Gamma$ is the gamma function. 

Using this representation, we can now compute the CVDs and CDDs directly for the entire distribution simultaneously, rather than from individual bins. This is essential for our application to uncertainty estimation, but may also prove useful in other contexts.


We first note the following two properties of gamma-distributed random variables: 

\begin{eqnarray}
a \mathbf{X} =& G(k,a \theta ) \\
\mathbf{X}^{-1} =& G^{-1}(k,\theta^{-1} )
\end{eqnarray}

where $G^{-1}$ is an inverse gamma distribution

\begin{eqnarray}
G^{-1}(x,k,\theta) =& \frac{\displaystyle x^{-k-1} exp(- \frac{\displaystyle \theta}{ \displaystyle k})}{\displaystyle \theta^{k} \Gamma(k)} 
\end{eqnarray}

From this is follows that the CVD $\mathbf{V_m}$ and CDD $\mathbf{\Delta_m}$ for a myelinated fibre bundle of length $L$, g-ratio $g$, and ADD scale and shape parameters $k_m$ and $\theta_{D_{m}}$, are given by 

\begin{eqnarray}
\mathbf{V_m} =& G(k_m, \theta_{V_{m}}) \\
\mathbf{\Delta_m} =& G^{-1}(k_m, \theta_{\Delta_{m}})
\end{eqnarray}

where 

\begin{eqnarray}
\theta_{V{_m}} =& g^{-1} \theta_{D_{m}}  \\
\theta_{\Delta{_m}} =& L^{-1} \theta_{V_{m}}
\end{eqnarray}

Finally, the mean $\mu_{\Delta_m}$ (or expectation $E_{\Delta_m}$) of $\mathbf{\Delta}_m$ is given arithmetically by 

\begin{eqnarray}
\mu_{\Delta_m} =& \frac{\displaystyle k_m}{\displaystyle \theta_{\Delta_m} -1} \end{eqnarray}

and the mode $M_{\Delta_m}$ by 

\begin{eqnarray}
M_{\Delta_m} =& \frac{\displaystyle k_m}{\displaystyle \theta_{\Delta_m} +1}
\end{eqnarray}

A summary of these calculations is shown in figure 1. 

