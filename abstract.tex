\begin{abstract}

Communication between spatially distant locations in the brain is mediated by long-range white matter fibre bundles, which vary considerably in their length, axonal diameter and myelin thickness distributions, as well as their relative proportions of myelinated vs. unmyelinated axons. These quantities are the principal determinants of the conduction delay distribution (CDD) associated with monosynaptic connections between a given pair of brain regions, which plays an important role in models of neural dynamics and information processing. With recent developments in MR acquisition and modelling techniques, it is now becoming possible to obtain estimates of several of the above variables \textit{in-vivo}, raising the exciting prospect of characterizing the CDD of a given white matter pathway directly from noninvasive structural measurements. The present study provides a novel technical contribution to this endeavour by undertaking a systematic uncertainty evaluation for microstructure-based conduction delay estimation, using linear error propagation techniques. Assessing the level of precision (inverse uncertainty) in model-derived estimates of an outcome variable (e.g. the parameters or moments of a CDD) requires not just the identification of the relevant variables (here we focus on tract lengths, axon diameters, and g-ratios), but also a quantification of the reduction in uncertainty in that outcome variable that their measurement provides. This depends on three factors: i) actual variability of the quantity of interest (within and across subjects) ii) accuracy of the measurement techniques, and iii) the relative contribution of each measured variable to the outcome variable, for a given measurement model. We introduce a novel measurement model for the CDD, by taking the standard formula for calculating an individual axon's conduction delay given its length, diameter, and g-ratio, and applying it directly to a gamma distribution of diameters through its scale parameter. The result is an inverse-gamma form for the CDD that has arithmetically computable moments, and is amenable to uncertainty evaluation using analytic techniques. Our results indicate that for myelinated fibre CDDs, axonal diameter distribution and myelin thickness parameters contribute substantially more to uncertainty in the CDD than tract lengths. Additionally, we observe that for 'complete' CDDs incorporating both myelinated and unmyelinated fibres, all other parameters are dwarfed by the influence of the mixing fraction for the two fibre types. Implications of these findings for use of conduction delays in neurophysiological models and in studies of white matter degeneration are discussed. These findings highlight the importance of methodological improvement and wider adoption of new microstructure imaging techniques for noninvasive characterization of conduction delays, and provide a framework for further enhancement and refinement of CDD estimates. 


\end{abstract}
