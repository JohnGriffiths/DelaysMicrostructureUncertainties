
\begin{eqnarray}
G(x,k,\theta) =& \frac{\displaystyle x^{k^{-1} exp(- \frac{\displaystyle x}{ \displaystyle \theta})}}{\displaystyle \theta^{k} \Gamma(k)} 
\end{eqnarray}

where $\Gamma$ is the gamma function. 

(We note that alternative and potentially superior parametric distributions have been proposed for the ADD (e.g. Sepehrband et al. \citep{sepehrband2016parametric}). Here we restrict ourselves to the gamma distribution since it has been the most widely used to date; however the majority of the following is easily adapted to alternative distribution functions.)



Using this representation, we can now compute the CVDs and CDDs directly for the entire distribution simultaneously, rather than from individual bins. This is essential for our application to uncertainty estimation, but may also prove useful in other contexts.


We first note the following two properties of gamma-distributed random variables: 

\begin{eqnarray}
a \mathbf{X} =& G(k,a \theta ) \\
\mathbf{X}^{-1} =& G^{-1}(k,\theta^{-1} )
\end{eqnarray}

where $G^{-1}$ is an inverse gamma distribution

\begin{eqnarray}
G^{-1}(x,k,\theta) =& \frac{\displaystyle x^{-k-1} exp(- \frac{\displaystyle \theta}{ \displaystyle k})}{\displaystyle \theta^{k} \Gamma(k)} 
\end{eqnarray}

From this is follows that the CVD $\mathbf{V_m}$ and CDD $\mathbf{\Delta_m}$ for a myelinated fibre bundle of length $L$, g-ratio $g$, and ADD scale and shape parameters $k_m$ and $\theta_{D_{m}}$, are given by 

\begin{eqnarray}
\mathbf{V_m} =& G(k_m, \theta_{V_{m}}) \\
\mathbf{\Delta_m} =& G^{-1}(k_m, \theta_{\Delta_{m}})
\end{eqnarray}

where 

\begin{eqnarray}
\theta_{V{_m}} =& g^{-1} \theta_{D_{m}}  \\
\theta_{\Delta{_m}} =& L^{-1} \theta_{V_{m}}
\end{eqnarray}

Finally, the mean $\mu_{\Delta_m}$ (or expectation $E_{\Delta_m}$) of $\mathbf{\Delta}_m$ is given arithmetically by 

\begin{eqnarray}
\mu_{\Delta_m} =& \frac{\displaystyle k_m}{\displaystyle \theta_{\Delta_m} -1} \end{eqnarray}

and the mode $M_{\Delta_m}$ by 

\begin{eqnarray}
M_{\Delta_m} =& \frac{\displaystyle k_m}{\displaystyle \theta_{\Delta_m} +1}
\end{eqnarray}

A summary of these calculations is shown in figure 1. 
