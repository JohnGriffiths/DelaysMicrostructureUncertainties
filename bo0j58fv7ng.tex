

\section{Discussion}

% 1. what we did
% 2. develop points
% 3. relevance to modeling
%    - fix delays
%    - significance of delay distribution (c.f. buzaki paper and other stuff you know)
%    - computational aspects
% 4. relevance to clinical
%    - ageing, MS, etc.; cognitive /  neural slowing 
%    - synchrony 
%    - (move some stuff from intro to here?)

% 5. caveats
%    - didn't consider coefficient
%    - maybe more correlations between variables
%    - don't know diameter distribution for unmyelinated
%    - not clear what significance of unmyelinated is and how it should be incorporated
%    - different distributions could be used
%    - often modelling studies sweep through delays 

% 6. future work
%     - more basic anatomical work needed to improve priors etc. 
%     - develop full atlases of delays from whole brain axcaliber data etc. 



% (For discussion? Recent work has suggested that other probability distributions such as X and Y provide a superior fit than the gamma distribution;  the methods discussed in the following )


The aim of present study was to assess the extent to which anatomically-based conduction delay estimates for long-range axonal fibre bundles are affected by uncertainty in their macrostructural and microstructural properties. In the interests of tractability and applicability we restricted our investigation to a relatively small set of variables that are known to have the most salient influence on the speed of axonal transmission, and that also present the possibility of measurement using emerging noninvasive imaging techniques. We anticipate that future work building on the framework developed here shall be both more expansive - through use of additional macro- and microstructural detail - and more focused, for example by examining more closely the influence of precision in delay parameterization for specific physiological and computational models, and in specific clinical contexts. In the following we briefly summarize our key results, highlight some important caveats, and offer some observations and suggestions as to how such future studies may proceed.


\subsection{Summary of key findings}

The central result of this paper is that is that, when considering myelinated fibres only, uncertainty in tissue microstructure properties (ADD parameters and g-ratios) contributes more to anatomically-based conduction delay estimates than uncertainty in axon or tract length. [ Another sentence here with specific numbers for this statement]. We emphasize that this conclusion is predicated on the assumptions we made regarding the uncertainty in length estimates; namely that it is very low. Note that this is not a statement about the role of length per se in conduction delays: clearly, a 180mm fibre bundle will have a substantially longer delays than a 20mm fibre bundle (see e.g. figure X). However, in the usual case where detailed maps of individual subjects' cortical geometry are readily available (through e.g. T1-weighted MRI scans), and particularly when diffusion-weighted data is available for tractography reconstructions, the degree of \textit{uncertainty} in tract length is not very high. We selected a 10mm uncertainty level heuristically based on general knowledge of white matter neuroanatomy in general and of tractography-based length estimates in particular. This is probably an overly-conservative estimate; it seems likely that properly applied multivariate and machine learning techniques should be capable of predicting tract lengths from T1-derived variables such as head size and shape with substantially greater accuracy. 

Secondly, we found a greater contribution to overall uncertainty from g-ratio than from the ADD shape parameter. [ Another sentence on details of this ]. This is perhaps surprising. Currently noninvasive g-ratio estimation using magnetization transfer imaging is less well-developed than ADD distribution estimation using AxCaliber. 

Third, we examined the case of CDDs resulting from mixtures of myelinated and unmyelinated fibres. Interestingly, this 'full CDD' case, as opposed to the partial CDDs provided by myelinated fibres only, has been largely ignored in computational studies to date investigating the physiological and computational significance of CDDs. similarly, in the present study we have chosen to focus on and emphasize the myelinated-only measurement models. The reasons for this are threefold: i) the mixed distribution involved additional quantities for which less information is available (mixing fraction and unmyelinated fibre ADD and conduction properties), ii) the role and significance of unmyelinated fibres is unclear, and arguably their influence is subtle and negligible, and iii) it is unclear how these properties might be measured with the noninvasive techniques about which the present study is primarily concerned with. Nevertheless, we suggest that greater consideration should be given to the potential role of unmyelinated fibres in CDDs and related phenomena, a proposal which is borne out by the results of our analyses. 

In the uncertainty evaluations using measurement model $f_3$, the propertion of myelinated vs. unmyelinated fibres had a dramatically higher influence on the CDD than any of the other variables considered. This is not surprising given the difference in the individual CDDs of the two fibre populations. We also emphasize that the outcome variable studied - expected value of $\Delta_{m+u}$ - is a rather blunt object for describing bimodal distributions of this kind. It is nevertheless highly informative in this case. 


