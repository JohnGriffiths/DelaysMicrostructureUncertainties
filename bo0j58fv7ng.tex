

\section{Discussion}

% 1. what we did
% 2. develop points
% 3. relevance to modeling
%    - fix delays
%    - significance of delay distribution (c.f. buzaki paper and other stuff you know)
%    - computational aspects
% 4. relevance to clinical
%    - ageing, MS, etc.; cognitive /  neural slowing 
%    - synchrony 
%    - (move some stuff from intro to here?)

% 5. caveats
%    - didn't consider coefficient
%    - maybe more correlations between variables
%    - don't know diameter distribution for unmyelinated
%    - not clear what significance of unmyelinated is and how it should be incorporated
%    - different distributions could be used
%    - often modelling studies sweep through delays 

% 6. future work
%     - more basic anatomical work needed to improve priors etc. 
%     - develop full atlases of delays from whole brain axcaliber data etc. 



% (For discussion? Recent work has suggested that other probability distributions such as X and Y provide a superior fit than the gamma distribution;  the methods discussed in the following )


The aim of present study was to assess the extent to which anatomically-based conduction delay estimates for long-range axonal fibre bundles are affected by uncertainty in their macrostructural and microstructural properties. In the interests of tractability and applicability we restricted our investigation to a relatively small set of variables that are known to have the most salient influence on the speed of axonal transmission, and that also present the possibility of measurement using emerging noninvasive imaging techniques. We anticipate that future work building on the framework developed here shall be both more expansive - through use of additional macro- and microstructural detail - and more focused, for example by examining more closely the influence of precision in delay parameterization for specific physiological and computational models, and in specific clinical contexts. In the following we briefly summarize our key results, highlight some important caveats, and offer some observations and suggestions as to how such future studies may proceed.


\subsection{Summary of key findings}

The central result of this paper is that is that, when considering myelinated fibres only, uncertainty in tissue microstructure properties (ADD parameters and g-ratios) contributes more to anatomically-based conduction delay estimates than uncertainty in axon or tract length. [ Another sentence here with specific numbers for this statement]. We emphasize that this conclusion is predicated on the assumptions we made regarding the uncertainty in length estimates; namely that it is very low. Note that this is not a statement about the role of length per se in conduction delays: clearly, a 180mm fibre bundle will have a substantially longer delays than a 20mm fibre bundle (see e.g. figure X). However, in the usual case where detailed maps of individual subjects' cortical geometry are readily available (through e.g. T1-weighted MRI scans), and particularly when diffusion-weighted data is available for tractography reconstructions, the degree of \textit{uncertainty} in tract length is not very high. We selected a 10mm uncertainty level heuristically based on general knowledge of white matter neuroanatomy in general and of tractography-based length estimates in particular. This is probably an overly-conservative estimate; it seems likely that properly applied multivariate and machine learning techniques should be capable of predicting tract lengths from T1-derived variables such as head size and shape with substantially greater accuracy. 

Secondly, we found a greater contribution to overall uncertainty from g-ratio than from the ADD shape parameter. [ Another sentence on details of this ]. This is perhaps surprising. Currently noninvasive g-ratio estimation using magnetization transfer imaging is less well-developed than ADD distribution estimation using AxCaliber. 

Third, we examined the case of CDDs resulting from mixtures of myelinated and unmyelinated fibres. Interestingly, this 'full CDD' case, as opposed to the partial CDDs provided by myelinated fibres only, has been largely ignored in computational studies to date investigating the physiological and computational significance of CDDs. similarly, in the present study we have chosen to focus on and emphasize the myelinated-only measurement models. The reasons for this are threefold: i) the mixed distribution involved additional quantities for which less information is available (mixing fraction and unmyelinated fibre ADD and conduction properties), ii) the role and significance of unmyelinated fibres is unclear, and arguably their influence is subtle and negligible, and iii) it is unclear how these properties might be measured with the noninvasive techniques about which the present study is primarily concerned with. Nevertheless, we suggest that greater consideration should be given to the potential role of unmyelinated fibres in CDDs and related phenomena, a proposal which is borne out by the results of our analyses. 

In the uncertainty evaluations using measurement model $f_3$, the propertion of myelinated vs. unmyelinated fibres had a dramatically higher influence on the CDD than any of the other variables considered. This is not surprising given the difference in the individual CDDs of the two fibre populations. We also emphasize that the outcome variable studied - expected value of $\Delta_{m+u}$ - is a rather blunt object for describing bimodal distributions of this kind. It is nevertheless highly informative in this case. 


\subsection{Implications for computational models}

% we focus on connectomics models

The anatomically-based CDD estimates discussed in this paper may be useful for a variety of clinical and research contexts. Our primary interest is for constraining macro-scale neural mass models. An exhaustive analysis of how CDD estimates may be useful in this context is beyond the scope of the present paper, but we may make some initial observations. Within the emerging paradigm in large-scale brain network modelling laid down by Ghosh et al. \citeyear{ghosh2008noise}, Deco et al. \citeyear{deco2009key}, Ritter et al. (2013), and others, it is typical to use a single conduction delay per connection, and to specify the value of that delay from fibre lengths from diffusion MR tractography, and a single global conduction velocity. The use of a single delay is a simplification of the CDD, and is best understood as representing its mean/mode. Thus measurement model $f_2$ is most relevant to such approaches. There are two chief ways in which microstructure-based delay estimation may prove useful for such studies. The first is as a constraint. It is common in modelling studies to perform \textit{parameter space exploration}, where conduction velocity and other parameters such as global coupling are varied throughout their entire range, and scalar summary metrics of the system's behaviour are plotted, for example as 2D heatmaps. A typical observation in PSE investigations is that conduction velocity changes stability surface but not overall structure (Knock et al. 2009; Jirsa book chapter). Delays also affect, and potentially create, oscillations; although how this interaction plays out is highly dependent on the specific model neural models used. Neural mass models generating strong local oscillations (e.g. alpha rhythms; Kunze et al. 2016) are less affected so by varying conduction velocity than models where oscillations emerge from large-scale circuits. For example, in the Robinson thalamocortical (refs), loop delay is v important. 
Something about delays in lefebvre models...
In principle, fixing conduction velocity from microstructure variables will allow this parameter to be fixed, thus allowing more flexibility in exploring and estimating other important parameters such as global gains and local excitatory-inhibitory interactions. Relatedly, some modeling approaches look to estimate conduction delays from neurophysiology data (Robinson model papers; DCM papers). This can be highly problematic (David refs). Microstructure data would be helpful here in constraining, and/or fixing, delay parameters. 
A second potentially important contribution of microstructure-estimated CDDs is that they have the potential to modify substantially the relative delays of network connections. In PSEs where conduction velocity is varied globally, the delays of each network connection may all increase and decrease, but their relative values (determined by tract lengths) remains constant. An important empirical question, for which there is currently very little relevant data in humans, is how much this relative delay structure, imposed essentially by inter-regional distances, is modified by per-tract differences in axonal calibers. 

A number of authors have studied neural mass and field models with distributed delays (e.g. Roberts \& Robinson R

%\citeyear{roberts 2009; Bojak \& Liley \citeyear{bojak2010axonal}; Hutt et al. 2006). Some sentences on this...


% the implicit assumption of using a single delay is that that is the most common delay or something like that
% ...which is what we're characterizing with the Expectation estimates

% 


\subsection{Implications for white matter degeneration}

In this study we have, for practical purposes, restricted our attention to non-disease questions of variability within the healthy population, and questions of measurement precision and the general relationship between delays and white matter microstructure in human and non-human brains. Naturally however, CDDs and conduction delays in general have major potential as a clinical biomarker, and to understand the implications of white matter pathology. The effect of disease-related loss of axons on the CDD depends on the whether the loss of axons is felt uniformly across the population, or whether it targets for example large fibres (ref) or small fibres (ref). If the effect is uniform, then the CDD should not change. In contrast, if there is specific loss to large fibres (above the mean; > X um), then it the effect would be to increase delay. If small, and/or unmyelinated fibres are targeted selectively, then (perhaps paradoxically) the result would be faster conduction on average. Reduction in myelin thickness would instead be expected to result in global slowing. There are multiple additional factors. However myelin thinning per se may not be a particularly common pathology. Myelin blistering, unwrapping, and then also remyelination (which also have been hypothesized to increase delays due to smaller internodes) may also have important effects on the delay structure of a given fibre bundle. The microstructure-based delay estimation framework used here would allow at least some of these various effects to be brought together (e.g. myelin and axonal changes), and also allows their importance to be assessed in relation to fibre length (slowing of long fibres is more significant than slowing of short fibres).  These changes may have influences on  neural synchrony, cognitive function, and coding.



In conclusion, we have presented an approach to characterizing and quantifying uncertainty in microstructure-based conduction delay estimates. Noninvasive microstructure imaging is a new and rapidly developing field, that has considerable potential for improving our understanding of brain dynamics in health and disease through both macroscopic and microscopic neurophysiological models. We anticipate that the future work shall build on and improve the estimates and recommendations made here, in concert with improvements in technology and understanding of brain organization in human, animal, and in-silico neural systems. 






%Furthermore, as noted above, the distribution of tract lengths for a given connection places an upper limit on the potential precision of summary statistics such as expected values when more coarse-grained parcellations are used.  distribution mean that lengths cannot be resolved. 
% tttttmpestimates such as the expected valueo f a delay sitribution. 
% delay estimates.

%Whilst this information has clear direct applications in models of neurodegenerative changes such as those occuring in advanced age, the primary intention of the evaluations in this chapter was to clarify what kinds of anatomical information are most useful for researchers and for the field as a whole to invest in acquiring (whether from imaging or from other sources), and to what degree of accuracy. A comprehensive assessment of this kind of question would consider many more factors and models than the simple delay distribution models described here. The approach described here could however easily be adapted, modified, or extended to address a broader set of questions. As such, it represents a generic tool for prospectively evaluating the utility of emerging imaging technologies in modelling research.



%\subsection{T1 proxy measures}



%In the comparison of non-tractography tract length estimators, a regression model based on Euclidean distances between regions was marginally superior to the more complex inverse warping approach, with prediction errors of \textasciitilde{}10mm. Note however that neither of these techniques were developed and optimized here with the full rigour of a pure methodological study, as that was not the purpose of this investigation. The aim was rather to identify the likely lower bounds on the degree of accuracy attainable without using tractography directly. The length prediction accuracy could most likely be improved considerably by both a more sophisticated predictive model and more sophisticated procedure for construction and inverse warping of atlas structures. The greatest improvements in the inverse-normalization procedure are likely to come from improved registration methods. Variants could explore different registration approaches, such as direct registration and warping of streamlines (e.g. Barrick \& Clark, 2004; Lawes et al., 2008), diffeomorphic methods, or more widely used volumetric methods (e.g. Yendiki et al., 2011; Clayden et al., 2011). The method described by Bojak et al. (2011), which identifies the shortest geodestic path betwen two ROIs within the white matter volume of a T1-weighted image, should also be considered in future comparisons. It would however be preferable to retain the methodological simplicity of the Euclidean distance regression approach, if possible. Future efforts may therefore be most efficiently spent constructing a large, high quality tractography training dataset, together with a more principled and systematic deployment of modern machine learning techniques.



%\subsection{Caveats}

%The conclusions from the uncertainty evaluation of this study must be understood with a number of caveats. Firstly, the simple phenomenological model for conduction delays ignores many of the details of real axonal conduction. The validity of a linear diameter-velocity relationship is not normally disputed however. The specific choices of the various components of the model may be more controversial. Researchers have disagreed in the past over the precise value of the constant of proportionality $c_{m}$ (see discussion in Bojak et al. \citeyearNP{bojak2010axonal}). Our choice of a gamma distribution function for the myelinated fibre diamters is not controversial, although the parametric form and moments of the unmyelinated fibre population were necessarily more speculative. Similarly, many of the uncertainties on parameter values specified in this study involved varying degrees of extrapoloation from published observations. This is particularly important, as the results of uncertainty evaluations are of course influenced considerably by the uncertainty levels specified. For example, the decision to define uncertainty levels in myelinated axon diameter distributions based on the range of values observed in figure 4.4 may be disputed. It may be preferred to instead treat these probabilities discretely, i.e. assuming that the `true' distribution for a given bundle must be one of the five, and assigning uncertainties according to relative confidences in each value. The measurement models used here may also be improved through more sophisticated specification of priors, and of the covariances between $g$, $D_{m}$, and $L$. The simplifying assumption of a single value for $L$ and $g$, as opposed to a distribution of $g$s and $L$s, could also be relaxed quite easily by increasing the uncertainty in these parameters in accordance with the dispersion of such distributions.



%\subsection{Relevance to neuroimaging}


%Previous studies have considered the computational, metabolic, and evolutionary advantages and implications of the relationship between anatomy and conduction delays. To date these studies have tended to focus on fairly gross differences between species, using ex vivo light and/or electron microscopy data, and small sample sizes. Whether the results of the present study prove useful in this line of research will depend on the tradeoff between the poorer data quality and resolution of imaging compared to microscopy on the one hand, and the benefits of being able to study individual differences and cognitive and clinical associations in humans, potentially with much larger a sample sizes, on the other.

%The groundwork of this study is important for the future work because it helps establish the current and near-future limits on the scope of tissue microstructure-based investigations into the effects of age on conduction delays. This relationship is an interesting one outright, but in the context of the present study it is developed primarily as a methodological tool for studying how disconnection-type scenarios, arising from several co-morbid forms of white matter degeneration (principally: axonal dystrophy, axonal loss, myelin dystrophy, myelin loss, shifted ratios of fast- and slow-conducting or myelinated- and unmyelinated fibres, in some as yet unknown combination), are related to the gradual cognitive decline generally seen in otherwise healthy older adults. Being able to distinguish between these various degenerative processes, and/or to quantify the incidence of each in individuals and groups of individuals, is an important basic science question that has relevance both to ageing per se and to several neurodegenerative diseases. Basic science is the focus here, however. 

%Later chapters in this thesis explore convergence between structural and functional data through the lens of dynamic connectivity and conduction delays.


%\subsection{Relevance to white matter disconnection and ageing research}


%Unlike the other chapters in this thesis, the studies described here have not been directly aimed at addressing questions relating, to $H_{wm}$, $H_{del}$, or other aspects of ageing or white matter disconnection. Rather, they provide important groundwork for the broader goal of bridging the gaps between brain structure, brain dynamics, and cognition. Precisely what level of microstructure information will be possible to obtain from new neuroimaging methods in the coming years remains unclear, and the analyses described were by design not overly specific on this matter. The general format (and code) for the uncertainty evaluations used here can, and are fully intended to be, adjusted in future studies and applied to specific use-cases and scientific questions. Certainly, in vivo structural measurements of white matter anatomy are envisaged as a key component of the new new physiologically-based disconnection framework advocated in Chapter 1, and developed in Chapter 3. As in all of science, a precise characterization of measurement and modelling uncertainties will be instrumental to progress in this area.





%\begin{figure}
%\centering
%\includegraphics[width=0.3\textwidth]{frog.jpg}
%\caption{\label{fig:frog}This frog was uploaded via the project menu.}
%\end{figure}

%\subsection{How to add Comments}

%Comments can be added to your project by clicking on the comment icon in the toolbar above. % * <john.hammersley@gmail.com> 2016-07-03T09:54:16.211Z:
%
% Here's an example comment!
%
%To reply to a comment, simply click the reply button in the lower right corner of the comment, and you can %close them when you're done.

%Comments can also be added to the margins of the compiled PDF using the todo command\todo{Here's a comment in %the margin!}, as shown in the example on the right. You can also add inline comments:

%\todo[inline, color=green!40]{This is an inline comment.}

%\subsection{How to add Citations and a References List}

%You can upload a \verb|.bib| file containing your BibTeX entries, created with JabRef; or import your \href{https://www.overleaf.com/blog/184}{Mendeley}, CiteULike or Zotero library as a \verb|.bib| file. You can then cite entries from it, like this: \cite{greenwade93}. Just remember to specify a bibliography style, as well as the filename of the \verb|.bib|.

%You can find a \href{https://www.overleaf.com/help/97-how-to-include-a-bibliography-using-bibtex}{video tutorial here} to learn more about BibTeX.

%We hope you find Overleaf useful, and please let us know if you have any feedback using the help menu above --- or use the contact form at \url{https://www.overleaf.com/contact}!



%\bibliographystyle{apacite} % JG_MOD {alpha}
%\bibliography{readcube_export}

