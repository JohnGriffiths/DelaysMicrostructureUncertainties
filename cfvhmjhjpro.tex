
\begin{abstract}

Communication between spatially distant locations in the brain is mediated by long-range white matter fibre bundles, which vary considerably in their length, axonal diameter and myelin thickness distributions, as well as their relative proportions of myelinated vs. unmyelinated axons. These quantities are the principal determinants of the conduction delay distribution (CDD) associated with monosynaptic connections between a given pair of brain regions, which plays an important role in models of neural dynamics and information processing. With recent developments in MR acquisition and modelling techniques, it is now becoming possible to obtain estimates of several of the above variables \textit{in-vivo}, raising the exciting prospect of characterizing the CDD of a given white matter pathway directly from noninvasive structural measurements. The present study provides a novel technical contribution to this endeavour by undertaking a systematic uncertainty evaluation for microstructure-based conduction delay estimation, using linear error propagation techniques. Assessing the level of precision (inverse uncertainty) in model-derived estimates of an outcome variable (e.g. the parameters or moments of a CDD) requires not just the identification of the relevant variables (here we focus on tract lengths, axon diameters, and g-ratios), but also a quantification of the reduction in uncertainty in that outcome variable that their measurement provides. This depends on three factors: i) actual variability of the quantity of interest (within and across subjects) ii) accuracy of the measurement techniques, and iii) the relative contribution of each measured variable to the outcome variable, for a given measurement model. We introduce a novel measurement model for the CDD, by taking the standard formula for calculating an individual axon's conduction delay given its length, diameter, and g-ratio, and applying it directly to a gamma distribution of diameters through its scale parameter. The result is an inverse-gamma form for the CDD that has arithmetically computable moments, and is amenable to uncertainty evaluation using analytic techniques. Our results indicate that for myelinated fibre CDDs, axonal diameter distribution and myelin thickness parameters contribute substantially more to uncertainty in the CDD than tract lengths. Additionally, we observe that for 'complete' CDDs incorporating both myelinated and unmyelinated fibres, all other parameters are dwarfed by the influence of the mixing fraction for the two fibre types. Implications of these findings for use of conduction delays in neurophysiological models and in studies of white matter degeneration are discussed. These findings highlight the importance of methodological improvement and wider adoption of new microstructure imaging techniques for noninvasive characterization of conduction delays, and provide a framework for further enhancement and refinement of CDD estimates. 

\end{abstract}

\section{Background}

\subsection{Conduction delays in neural and cognitive function}




Conduction delays play a key role in a number of computational and dynamic aspects of brain function, as well as placing certain constraints on its structural and functional organization. For example, it has been proposed that time delays in local circuits facilitate complex computations in minimal time \cite{izhikevich2006polychronization}, and that relative spike timings represent a plausible coding principle for rapid and reliable visual object recognition \cite{vanrullen2002surfing}. Precise coordination of action potential arrival times at the sub-millisecond scale are also known to be critical for oscillatory synchrony \cite{fries2005a}, and for spike-time dependent plasticity 
\cite{markram1997regulation, zhang1998a}; although the prevalence of temporal (as opposed to rate) coding in the majority of cortical function has been challenged on theoretical grounds \cite{shadlen1998the}. Delays also play a key role in sculpting the dynamics of neural systems at the coarser meso- or macroscopic spatial and temporal scales studied in humans with noninvasive techniques such as electroencephalography (EEG), magnetoencephalography (MEG), and functional magnetic resonance imaging (fMRI). Models of the EEG alpha rhythm generation as reflecting global modes in corticocortical \shortcite{nunez2006electric}  or thalamocortical \shortcite{robinson2003neurophysical} networks consider the location of peaks in the M/EEG power spectrum to be due to characteristic time delays in long-range fibre bundles. Ghosh and colleagues (Ghosh et al. \citeyearNP{ghosh2008noise}; see also Deco et al. \citeyearNP{deco2009key}) studied the effects of delays on dynamics in networks of coupled oscillators with realistic cortical topologies and distance-dependent delays. In their study, a model with time delays was found to reproduce several characteristic features of M/EEG and fMRI data: intermittent high-frequency oscillations with transient and spindle-like time courses and dominant frequencies in the alpha range (8-12Hz), and low-frequency (<0.1Hz) oscillations in regions of parietal and cingulate cortices, closely matching the classic default-mode network patterns.  Ghosh et al. state that “The space-time structure of the couplings defined by the anatomical connectivity (space) and the time delays (time) will be the primary component shaping
the dynamic repertoire of any large scale network”. These observations strongly suggest that delay changes due to white matter degeneration could have potentially extensive effects on neural and cognitive function. 

Limited support for a role of conduction delays (qua conduction velocities) in individual differences in cognitive abilities comes from the work of Vernon and colleagues, who found associations between a general intelligence factor and conduction velocity (McRorie \& Cooper  \citeyearNP{mcrorie2004synaptic}, Reed et al. \citeyearNP{reed2004confirmation}; Reed \& Jensen \citeyearNP{reed1991arm,reed1992conduction,reed1993a}
Vernon \citeyearNP{vernon1983speed}; Vernon \& Mori \citeyearNP{vernon1992intelligence} ; but see Wickett et al. \citeyearNP{wickett1994peripheral} for failed replications, and Saint-Amour et al. \citeyearNP{saintamour2005can} for criticisms). In ageing, reduced conduction velocities have been demonstrated in motor pathways of both aged cats \shortcite{morales1987basic,xi1999changes} and aged rats \shortcite{astonjones1980brain}. Evidence for central nervous system slowing in humans comes from the work of Allison and colleagues \shortcite{allison1984developmental,allison1983brain}, who studied the effects of age and sex on evoked potentials measured by EEG. They found that P50, P100, and P300 potentials showed an increase in latency with age.  Amplitude effects were also observed, but largely independent of latency changes. Whilst they list a number of potential causes of the effects, the authors favour slowed axonal conduction, possibly with a selective loss of large myelinated fibres \shortcite{morrison1990aging}, as the primary mechanism of these changes. Other authors have reported similar results (e.g. Dustman et al. \citeyearNP{dustman1990age}, Kuba et al. \citeyearNP{kuba2012aging}). More recently, in a series of studies on lifespan changes in EEG properties in a large cohort of 1400 subjects, Kerr et al. (\citeyearNP{kerr2010age} 
; see also Kerr et al. \citeyearNP{kerr2011modelbased}, Chiang et al. \citeyearNP{chiang2011age}, Van Albada et al. \citeyearNP{vanalbada2010neurophysiological}) reported an age-related increase in auditory oddball latency of ∼2ms/year. This body of work is notable because it exploits a sophisticated thalamocortical model of the neural dynamics that generate the signals measured by EEG. Importantly, this allowed the conclusion of an age trend in thalamocortical conduction delay. Complementary analyses of EEG spectra from the same subjects using the same generative biophysical model drew similar conclusions of increased thalamocortical delays in older subjects \shortcite{vanalbada2010neurophysiological}. Thus these authors echo Allison et al.  \citeyear{allison1984developmental} in concluding that an age-related change in the conduction velocity of white matter pathways underlie the effects seen in the EEG. 

The research summarized above gives gives some indication of the importance of conduction delays as an object of investigation in computational, cognitive, and clinical neuroscience. A comprehensive review is beyond the scope of this paper; further discussion can be found in Griffiths \citeyear{griffiths2014the}. We now review briefly several technical considerations of central importance to the present study: models, assumptions, and associated uncertainties involved in estimation of conduction delays from microscopy and non-invasive microstructure imaging data. 

