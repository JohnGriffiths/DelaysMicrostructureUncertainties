
%The level of uncertainty in the final result is $\textit$ - around time more sensitive to the level of uncertainty in g ratio than in $\( \theta_{m} \)$, and ~15 times more sensitive to the level of uncertainty in g ratio than in length. 
%} \label{fig:5}
%\end{figure}

%the myelinated fibre delay distribution shape parameter $\theta_{m} '$ mode of the myelinated+unmyelinated delay distribution $E(\Delta_{m+u}$) are shown in figures 4.5 and 4.6, respectively. 

%It should also be borne in mind that the use of a gamma form for the distribution of axon diameters is an approximation. Skewed- or log-normal distributions would also have the adequate form, or alternatively a Gaussian fit to the logarithm of axon diameters (c.f. Marner et al. \citeyearNP{marner2003marked}).

Results from uncertainty evaluations for model $f_1$ and model $f_2$ are shown in Figures 4 and 5, respectively. In these figures, combined standard uncertainties for different levels of uncertainty in the input parameters are presented as symmetric error bars above and below the expected value, which was always the same. These plots are ordered left-to-right from zero to maximum uncertainty. The most prominent feature of Figure 4 is the step-change that results from the addition of uncertainty in the myelinated fibres diameter
parameter, $\theta_{D_m}$ (compare data points 1-4 vs datapoints 5-8). Uncertainty in $L$ and $g$ each contribute \textasciitilde{}0.35 to the uncertainty in the measurand $\theta_{\Delta_m}$, whereas uncertainty in $\theta_{D_m}$ contributes 10 times this amount, so that the final estimate of $\theta_{\Delta_m}$ with maximal uncertainty is 4.1+/-3.0856. Accurate knowledge of the distribution of myelinated fibre diameters would therefore appear to be significantly more important than knowledge of fibre length or g-ratios in this case.

As the sensitivity coefficients in Figure 4 show, the computation of $\theta_{\Delta_m}$ is actually more sensitive to variation in $g$ than in $\theta_{D_m}$. However because the value of $g$ is generally observed to be quite stable \shortcite{rushton1951a}, it was assigned lower uncertainty in this analysis. The dominance of $\theta_{D_m}$ in Figure 4 is due to its relatively higher uncertainty, which was chosen based on the range of diameter distributions observed in post-mortem corpus callosum samples. The contribution of $L$ to the combined uncertainty is roughly equivalent to that of $g$, despite having considerably lower sensitivity coefficients. Nevertheless, given the order of magnitude difference in contribution to the combined uncertainty compared to $\theta_{D_m}$, the improvement in precision of using tractography vs. using a length proxy such as Euclidean distance with \textasciitilde{}10mm error appears to be negligible.