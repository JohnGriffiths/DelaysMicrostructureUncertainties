\subsection{Relative contributions of length, diameter, and myelin thickness to precision of conduction delay estimates}

The uncertainty evaluation technique outlined above provides a framework for assessing quantitatively and comprehensively the sensitivity to input parameters demonstrated visually and intuitively in the previous section. 

%Figure X shows sensitivity coefficients measurement models f1 and f2. 
%Figure X shows combined standard uncertainties for measurement models f1 and f2.
% (what does this look like?)
%Sensitivity coefficients for g and theta are higher than for L. 
% have 2D plot: uncertainty in outcome as a function of uncertainty level of theta and g (assume L is fixed)
% - g vs. theta which contributes more...
% - pdf variability plot for theta model


\emph{Uncertainty evaluation results for model $f_1$.}


\begin{figure}[h!]
\begin{center}
\includegraphics[width=10cm]{Chapters_1_2_3_4_5_6_7_711_1.png}% This is a *.jpg file
\end{center}
\begin{center}
\includegraphics[width=10cm]{Chapters_1_2_3_4_5_6_7_711_2.png}% This is a *.jpg file
\end{center}

\caption[\emph{Uncertainty evaluation results for model $f_1$.}]. Uncertainty evaluation results for model $f_1$. The step change after the first four points is due to the addition of uncertainty in $\theta_{D_m}$. Bottom panel shows sensitivity coefficients for the three variables with nonzero uncertainty.
\end{figure}

