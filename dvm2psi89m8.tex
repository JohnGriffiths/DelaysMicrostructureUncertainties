
\subsection{Implications for computational models}

% we focus on connectomics models

The anatomically-based CDD estimates discussed in this paper may be useful for a variety of clinical and research contexts. Our primary interest is for constraining macro-scale neural mass models. An exhaustive analysis of how CDD estimates may be useful in this context is beyond the scope of the present paper, but we may make some initial observations. Within the emerging paradigm in large-scale brain network modelling laid down by Ghosh et al. \citeyear{ghosh2008noise}, Deco et al. \citeyear{deco2009key}, Ritter et al. (2013), and others, it is typical to use a single conduction delay per connection, and to specify the value of that delay from fibre lengths from diffusion MR tractography, and a single global conduction velocity. The use of a single delay is a simplification of the CDD, and is best understood as representing its mean/mode. Thus measurement model $f_2$ is most relevant to such approaches. There are two chief ways in which microstructure-based delay estimation may prove useful for such studies. The first is as a constraint. It is common in modelling studies to perform \textit{parameter space exploration}, where conduction velocity and other parameters such as global coupling are varied throughout their entire range, and scalar summary metrics of the system's behaviour are plotted, for example as 2D heatmaps. A typical observation in PSE investigations is that conduction velocity changes stability surface but not overall structure (Knock et al. 2009; Jirsa book chapter). Delays also affect, and potentially create, oscillations; although how this interaction plays out is highly dependent on the specific model neural models used. Neural mass models generating strong local oscillations (e.g. alpha rhythms; Kunze et al. 2016) are less affected so by varying conduction velocity than models where oscillations emerge from large-scale circuits. For example, in the Robinson thalamocortical (refs), loop delay is v important. 

Something about delays in lefebvre models...
In principle, fixing conduction velocity from microstructure variables will allow this parameter to be fixed, thus allowing more flexibility in exploring and estimating other important parameters such as global gains and local excitatory-inhibitory interactions. Relatedly, some modeling approaches look to estimate conduction delays from neurophysiology data (Robinson model papers; DCM papers). This can be highly problematic (David refs). Microstructure data would be helpful here in constraining, and/or fixing, delay parameters. 
A second potentially important contribution of microstructure-estimated CDDs is that they have the potential to modify substantially the relative delays of network connections. In PSEs where conduction velocity is varied globally, the delays of each network connection may all increase and decrease, but their relative values (determined by tract lengths) remains constant. An important empirical question, for which there is currently very little relevant data in humans, is how much this relative delay structure, imposed essentially by inter-regional distances, is modified by per-tract differences in axonal calibers. 

A number of authors have studied neural mass and field models with distributed delays (e.g. Roberts \& Robinson 2009; Bojak & Liley 2010; Hutt et al. 2006). 

%\citeyear{roberts 2009; Bojak \& Liley \citeyear{bojak2010axonal}; Hutt et al. 2006). Some sentences on this...


% the implicit assumption of using a single delay is that that is the most common delay or something like that
% ...which is what we're characterizing with the Expectation estimates

% 