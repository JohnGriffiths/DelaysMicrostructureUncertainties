
\section{Methods}

\subsection{Uncertainty Evaluation}


In order to properly quantify the contribution of various sources of uncertainty in the CDD model outlined above, we conducted a series of formal uncertainty evaluations, in accordance with the rules and guidelines set out by the Joint Committee for Guides in Metrology (2008). An uncertainty evaluation begins by defining a function $f$, the \textit{measurement model}, that relates the \textit{measurand} (quantity of interest or outcome variable) $Y$ to the input quantities $X_{i}$; i.e. $Y = f(x_{1},...x_{n})$. Note that in our case the measurement model does not refer to the MRI microstructure or microscopy measurements process, but rather the computation of the CDD and derivative quantities given estimates of the various parameters discussed in the previous section. Once the measurement model is specified,
an uncertainty evaluation proceeds by obtaining expectations of the
probability distributions on the input quantities $X=X_{1}...,X_{N}$,
and the \emph{standard uncertainty}, or error, in those estimates. Expectations of the measurand $Y$ are then calculated straightforwardly from the measurement model and the expectations of $X$. The critical
part of the evaluation procedure is calculation of the standard
uncertainty of $Y$, which uses the mathematical technique of \emph{uncertainty} or \emph{error propagation}. For linear models this essentially consists of a weighted combination of the uncertainties of $X$, correcting for non-independence between inputs using covariance terms. Depending on the complexity of the model, this can be done using
analytically exact methods, analytic approximations, or sampling techniques such as Markov Chain Monte Carlo. The measurement models specified here were simple enough for an exact approach, which was implemented with custom error propagation routines modified from Rocklin et al. (\cite{rocklin2017symbolic}), using the python symbolic computing library \textit{sympy} \cite{meurer2017sympy}. The uncertainty evaluation procedure concludes by summarizing the results of uncertainty propagation. In addition to the standard uncertainty in $Y$, the model is summarized by the \emph{fractional} or \emph{relative uncertainty} $E/S$ (measurement of uncertainty divided by measured value) and \emph{expanded uncertainty} $FU=S/C$, which is the standard uncertainty multiplied by a coverage factor (an interval over which the majority of potential values of the measurand reside). Finally, \emph{sensitivity coefficients} $s_{1}...,s_{N}$ can be derived for each variable $X_{i}$ in the measurement model as the first order partial derivatives of $f$ with respect to $X_{i}$. These quantify the amount of influence that uncertainty in that variable has on the final estimate of the measurand.



\subsection{Measurement Models}

\subsubsection*{\textit{Model 1: Myelinated fibre delay distribution parameters}}

For the first set of uncertainty evaluations we define the measurand to be the myelinated fibres CDD scale parameter $\theta_{\Delta_m}$. Following eqs. 8-10, the measurement model is therefore given by $f_{1} = \theta_{\Delta_m} = L^{-1} g^{-1} \theta_{D_m}$. This provides a
`holistic' quantification of uncertainty in the entire CDD, since together with the shape parameter $k_{m}$ (which, conditional on $\theta_{m}$, has no additional uncertainty; see below) it defines the probability weighting of any delay lying within the support of $\Delta_{m}$. 


\subsubsection*{\textit{Model 2: Expected value of myelinated fibre delay distribution}}

Whilst $f_{1}$ provides an elegant and simple measurement model, it suffers from the problems that a) it is in units that have no direct physical or physiological interpretation, and b) the uncertainties propagated up from the model parameters are spread over the entire distribution in a somewhat unintuitive fashion. We therefore chose as the second measurement model the expected value or mean of the myelinated fibre CDD, as given by eq. 11: $f_2 = E_{\Delta_m} = \frac{k}{1-  k^{-1} (L^{-1} g^{-1} \theta_{D_m} - 1)}$. This is perhaps the most directly useful delay variable considered in this paper, as the overwhelming majority of modelling studies examining the effects of conduction delays use only a single delay for the connection between a given pair of brain regions, rather than multiple delays or full delay distributions. The use of a single delay can be understood as an approximation of the full delay distribution by its average (mean) or most common value (mode). We return to the issue of single vs. multiple/distributed delays in the discussion. 

%Range of values over which to vary delays over which microstructure-constrained delays may be varied?


\subsubsection*{\textit{Model 3: Expected value of mixed myelinated + unmyelinated delay distribution}}

%The first and second measurement models are based directly on eqs. 8-10, and use as measurands  $\theta_{{D_m}}$ and $\mu_{D_{m}}$, respectively. 

%Model 1 therefore quantifies, in a holistic fashion, uncertainty in the overall shape of the CDD. Model 2, in contrast, focuses on the expectation or central tendency of the CDD. This can be understood as giving a single, representative number for the conduction delay of a given fibre pathway, which is commonly done for the sake of simplification in modelling studies, 


%The first measurement model defines the measurand to be the myelinated fibres delay distribution ($\Delta_{m}$) scale parameter $\theta_{m} '$, making the measurement model $f_{1} = \theta_{m} ' = \frac{Lg\theta_{m}}{c_{m}}$. 


%I consider
%here two sets of measurement models based on the CDD concept discussed above. 

%In that chapter the myelinated and unmyelinated fibre delay distributions $\Delta_{m}$ and $\Delta_{u}$ were computed by sampling from gamma distributions of axon diameters and applying a simple transform to compute the delay for a given axon given its diameter. These were then pooled with relative proportions of samples $p_{m}$ and $p_{u} = 1-p_{m}$ to give a sampling-based mixture distribution $\Delta_{m+u}$. Here I use an alternative parameterization of the delay distributions, which uses an analytic rather than a sampling-based procedure for the computation of $\Delta_{m}$ and $\Delta_{u}$, and so is more amenable to analysis using linear error propagation techniques. The key difference is that the expressions for conductions for delays as a function of diameter $d$ are applied directly to the scale parameter of the fibre diameter distributions. The result is a mixture of \emph{inverse} gamma
%distributions; $\Delta_{m+u} = p_{m} \Gamma^{-1}(k_{m},\frac{Lg\theta_{m}}{c_{m}}) + p_{u} \Gamma^{-1}(k_{u}\frac{L}{\theta_{u}c_{u}})$.

Unmyelinated fibres are largely ignored in the microscopy and emerging neuroimaging literature on ADDs, primarily for the understandable reason that they are extremely difficult to measure from accurately due to their small size (mostly <1$\mu$m), which makes them effectively invisible to both light microscopy and MRI microstructure imaging.  Because electron microscopy is impossible in post-mortem human tissue due to the need for toxic tracer injections before death, estimates of unmyelinated fibre densities and diameters need to be extrapolated from studies in non-human primates such as the macaque. Importantly, such studies have observed as high as 40\% of axons to be unmyelinated in some brain locations \citep{bowley2010age}. A 'complete' CDD would therefore include contributions from both myelinated and unmyelinated fibres, and it is immediately evident that this could have a substantially longer tail than the CDD for myelinated fibres alone, given the substantially slower conduction velocities of unmyelinated fibers. We therefore included as our third measurement model the expectation of the mixed myelinated+unmyelinated fibre delay distribution; $f_{3} = E_{\Delta_{m+u}} = E(p_m  \mathbf{\Delta}_{m} + (1-p_m) \mathbf{\Delta}_{u}) $, where $p_m$ is the proportion of fibres in the fascicle that are unmyelinated. 

\textit{++TO DO: define $\Delta_u$}


% NEED TO DEFINE E_D_M+U...


%The procedure detailed above provides a general framework for assessing how uncertainty in length, g-ratio, and axon diameter distribution parameters impact on the precision on conduction delay estimates. As is normally the case with scientific measurements, these uncertainties have multiple origins. In the present case these can be separated first and foremost into within-subject variability, between-subject variability, and measurement error. The importance of the former depends on the magnitude of the latter: measurements are accurate only insofar as they increase the confidence in the estimate of a quantity beyond that of an estimate based only on the population variation. For the case of length, both the tractography and the T1 proxy measurements have relatively high accuracy, and so we can focus on the difference in precision between the two. For $g$ and $\theta_{m}$, the accuracy of the measurement techniques is lower, and so we are concerned more with the comparison of measurement vs.~no measurement at all. Parallelling this, the pragmatic reality is that neuroimaging studies at present collect only T1-weighted images as standard. Diffusion-weighed images for tractography are reasonably common, but the multiple b-value diffusion-weighted sequences necessary for axon diameter imaging, and the magnetization transfer sequences necessary for g-ratio imaging, remain highly specialist technologies. Thus the question for noninvasive estimates of length is one of `which', whilst for g-ratio and axon diameters is still one of `whether'. The uncertainty values compared in this study were chosen to reflect this situation.


\subsection{Parameter values}

The three measurement models $f_{1}$, $f_{2}$, and $f_{3}$ share input parameters
$L$, $\theta_{m}$, $k_{m}$, $g$, and $c_{m}$, with $f_{3}$ additionally
having parameters $\theta_{u}$, $k_{u}$, and $p_{m}$. None of these may
be said to be known perfectly and entirely without uncertainty. We restrict our focus in this study to uncertainty in parameters whose precision may be actually or potentially be improvable in the near future using noninvasive imaging: $L$, $g$, $\theta_{m}$, $p_{m}$. Each of these are considered with and without measurement uncertainty, and error propagation calculations were made for all permutations of zero and non-zero uncertainties in these parameters. The remaining parameters were set as constants. For all parameters apart from tract length $L$ (see below), zero uncertainty level should be understood as an idealization that is useful in understanding their respective contributions to the combined standard uncertainty. Non-zero levels were chosen to reflect the likely range of values which the parameter may take, considering both inter-regional and inter-subject variation.


\subsubsection*{\textit{Tract Length}}

The lengths of white matter fibre tracts vary widely in humans, from around 10mm to as high as 200mm in large brains. However, in contrast to the other model parameters specifying ADDs and myelination, it is not necessary to incorporate this full range of variation into the measurement model, because in practice additional information is available that substantially reduces uncertainty about tract length. In particular, diffusion MRI tractography allows fairly unambiguous measurement of tract lengths to a high degree of accuracy. Tract lengths can be also approximated to fairly high precision (10mm error) without use of tractography, simply using using Euclidean distance. It should be noted however that, depending on how the brain regions for which delay is being estimated are defined, there may be a genuine distribution of tract lengths associated with a given anatomical connection. Naturally, the width of this distribution will be larger when larger regions are used. We therefore set $L$ = 160mm+/- 20mm.

%uncertainty in length to X, for reason X (Euclidean distance approximation / length dispersion).


\subsubsection*{\textit{Axon diameters}}

Uncertainties for $\theta_{m}$ were based on the data reported in Aboitiz et al. (\cite{aboitiz1992fiber}). These authors used light microscopy to examine the five main segments of the corpus callosum (genu, anterior body, mid body, posterior body, and splenium). They presented the diameter distributions of each segment as histograms (figure 4 of that paper). To identify the physiological range of $k_{m}$ and $\theta_{m}$, the parameters of five gamma distributions were fitted to the histogram data presented in that paper (figure 4.4, left panel). This yielded a range of 2-5 for 
$\theta_{m}$, and so in this study $\theta_{m}$ was set to lie in the middle of this range, with an uncertainty that spans the majority of this range; 3.5+/-2. Because $\theta_{m}$ and $k_{m}$ are highly correlated, the appropriate value of $k_{m}$ was selected by regressing $k_{m}$ onto the $\theta_{m}$ (figure 3, right panel), and
using the regression model to specify the corresponding $k_{m}$ for a given $\theta_{m}$. Thus, for example, $\theta_m$=3.5 specified $k_m$=4.12.



