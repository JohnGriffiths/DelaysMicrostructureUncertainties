Fits to histograms of axon diameter counts indicate that myelinated fibre ADDs are well-described by a gamma distribution
\footnote{(We note that alternative and potentially superior parametric distributions have been proposed for the ADD (e.g. Sepehrband et al. \citep{sepehrband2016parametric}). Here we restrict ourselves to the gamma distribution since it has been the most widely used to date; however the majority of the following is easily adapted to alternative distribution functions.)}, with peak (mode) around 1$\mu$m and a long tail extending up to approximately 10$\mu m$. \citep{aboitiz1992fiber}. Several studies have used these microscopy-derived ADD histograms, together with the above equations, to obtain estimates the \textit{conduction velocity distribution} (CVD) and \textit{conduction delay distribution} (CDD) associated with a given white matter fascicle. This is done by computing Eq. 2 for all diameters within the ranges measured, and and assigning the CVD and CDD bin weightings directly from the corresponding ADD bin weightings. Whilst this approach of treating each histogram bin separately is the most empirically accurate and assumption-free way of computing CDDs (when an empirical ADD histogram is available), it is problematic when we want to know about uncertainty. This is due to the non-independence of nearby bins in the axon size count histograms. If, for example, we calculated variability (e.g. standard deviation over samples) in histogram bin weightings for different sections of the corpus callosum, and summed this uncertainty over bins, we would get a massively over-inflated estimate of the resultant total uncertainty, because adjacent bin weights will be highly correlated and therefore do not contribute independently to the overall uncertainty. 
