

%   \begin{figure}[htbp!]
%centering




    
%   \begin{center}
%   \adjustimage{max size={0.99\linewidth}{0.99\paperheight}}{Figures/Chapters_1_2_3_4_5_6_7_711_1.png}
%   \end{center}
    
    
%   \begin{center}
%   \adjustimage{max size={0.99\linewidth}{0.99\paperheight}}{Figures/Chapters_1_2_3_4_5_6_7_711_2.png}
%   \end{center}
    
    
%\caption[\emph{Uncertainty evaluation results for $\theta_{m}'$ .}]{\emph{Uncertainty evaluation results for $\theta_{m}'$ .}
%Top panel shows combined standard uncertainties. The step change after the first four points 
%is due to the addition of uncertainty in \( \theta_{m} \). 
%Bottom panel shows sensitivity coefficients for the three variables with nonzero uncertainty. 
%The level of uncertainty in the final result is ~3 times more sensitive to the level of uncertainty in 
%in g ratio than in \( \theta_{m} \), and ~15 times more sensitive to the level of uncertainty 
%in g ratio than in length. 
%}
%\label{fig:\emph{Uncertainty evaluation results for $\theta_{m}'$ .}}
%\end{figure}


\subsection{Myelinated/Unmyelinated fibre ratio dominates in combined models}



Combined standard uncertainties and sensitivity coefficients the from second uncertainty evaluation for the expectation of the mixed delay distribution, $E(\Delta_{m+u})$, are shown in figures 5 and 6, respectively. In this analysis, the parameters of the unmyelinated fibre population were considered constant, and so the only parameter that differed from the results for $\theta_{\Delta_m}$ presented above was the proportion of myelinated fibres, $p_{m}$. Interestingly, it is this parameter that completely dominates both the standard uncertainty and sensitivity coefficient results for $E(\Delta_{m+u})$. The y axis in Figure 5 is plotted on a logarithmic scale due to the order of magnitude differences between the sensitivity coefficients of $p_{m}$ (105.6) and of $L$ (0.2), $\theta_{D_m}$ (0.4), and $g$ (1.8). This dominance of $p_{m}$ is also evident in the step-change in combined uncertainties (first 8 vs.  second 8 data points in Figure 6), similar to that seen for $\theta_{D_m}$ in Figure 4. Introducing a modest level of uncertainty in $p_{m}$ increases the uncertainty in $E(\Delta_{m+u})$ from \textasciitilde{}22+/-2ms to \textasciitilde{}22+/- 10ms.










%\begin{landscape}

%\vspace{30 mm}
                

%    \begin{figure}[htbp!]
%centering


%   \begin{center}
%   \adjustimage{max size={0.99\linewidth}{0.99\paperheight}}{Figures/Chapters_1_2_3_4_5_6_7_724_1.png}
%   \end{center}
    
%\caption[\emph{Combined standard uncertainties for $E(\Delta_{m+u})$ .}]{\emph{Combined standard uncertainties for $E(\Delta_{m+u})$ .}
%Shown are combined standard uncertainties in \( E(\Delta_{m+u}) \), 
%ad in top panel of figure 4.5. A step change in combined standard uncertainty is seen when 
%uncertainty in the mixing proportion \( p_{m} \) is introduced. 
%}
%\label{fig:\emph{Combined standard uncertainties for $E(\Delta_{m+u})$ .}}
%\end{figure}

    
%                \end{landscape}
                

%    \begin{figure}[htbp!]
%centering

   
%   \begin{center}
%   \adjustimage{max size={0.99\linewidth}{0.99\paperheight}}{Figures/Chapters_1_2_3_4_5_6_7_726_1.png}
%   \end{center}
    
    
%\caption[\emph{Sensitivity coefficients for $E(\Delta_{m+u})$.}]{\emph{Sensitivity coefficients for $E(\Delta_{m+u})$.}
%Note that sensitivity coefficients are plotted on a log scale in this figure, due to the difference 
%in magnitude between the coefficients for \( p_{m} \) and the other three variables. Here, the level 
%of uncertainty in \( E(\Delta_{m+u} ) \) is roughly 60 times more sensitive to the level of 
%uncertainty in \(p_{m}\) than in g-ratio, and roughly 500 times more sensitive to the level of 
%uncertainty in \(p_{m}\) than in length. 
%}
%\label{fig:\emph{Sensitivity coefficients for $E(\Delta_{m+u})$.}}
%\end{figure}

