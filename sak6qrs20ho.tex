
\subsection{Present Study}

The principal aim of the present study is to examine the how estimates of various quantities relating to$\mathbf{\Delta}_m$ - chiefly $\theta_{\Delta_m}$ and $M_{\Delta_m}$, depend on uncertainty in the values of $L$,$k$, and $\theta_m$, and $g$, given the both the typical ranges of these parameters and the estimates that can potentially be obtained from noninvasive imaging. Our focus here is loosely on MR imaging in humans, but the general conclusions and techniques used may also be applied in microscopy studies (e.g. Caminiti et al. \citeyear{caminiti2009evolution}, Innocenti et al. \citep{innocenti2010fiber}) in both human and non-human samples. In the following sections we now describe in detail the uncertainty evaluation procedure, as well as the ranges and expected values chosen for the key parameters. 


% - lengths. a lot of work just using lengths and assuming a value for velocity. 
% - now we can estimate diameters and g-ratios. 
% - there are limitations in these measurements but in principle they can be used to make calculations of the kind done above. this is what we're concerned with here.
 
%Conduction delays play a key role in sculpting patterns of brain dynamics at the macroscopic scale (Deco et al., 2009). Until relatively recently it has been necessary for researchers looking to model these phenomena to use heuristics such as head size or population averages from post-mortem studies (Nunez \& Srinivasen 2006) to estimate the length of fibre pathways in individual subjects. Nowadays, diffusion-weighted MR tractography provides a more direct and noninvasive measure of this anatomical feature, and a number of researchers have incorporated tractography-based connection length values into their models (Izhikevich \& Edelman, 2008; Valdes-Sosa et al., 2009). Many still use non-tractography proxy methods, however, based on e.g. Euclidean distance (e.g.~Ghosh et al., 2008; Deco et al., 2009) or volumetric segmentation of T1-weighted MR images (Bojak et al., 2011). As a large proportion of this research is concerned with developing either forward (e.g. Bojak et al., 2011) or inverse (e.g. David et al. 2006; Valdes-Sosa et al. 2009) models that capture subject-specific anatomy and neural activity patterns, the relative accuracy of these various non-tractography or non-subject-specific proxy measures of fibre length is an important practical question. Similarly, a question that is certain to receive growing attention in the coming decade is how the `new breed' of MRI-based tissue microstructure metrics might contribute to this enterprise. Recent developments in structural MR imaging are making measurements of both axon diameters (Assaf et al., 2006, 2007, 2008; Alexander et al., 2010), myelin thickness (Stikov et al., 2011), myelin dystrophy (Avram et al., 2010). Whether these developments play a peripheral or a central role in future models of human brain dynamics will be a function of the precision of the measurements and the degree of precision required in the modelling process.



%\subsection{Conduction delays and neuroanatomy}

%Conduction delays play a key role in sculpting patterns of brain dynamics at the macroscopic scale (Deco et al., 2009). Until relatively recently it has been necessary for researchers looking to model these phenomena to use heuristics such as head size or population averages from post-mortem studies (Nunez \& Srinivasen 2006) to estimate the length of fibre pathways in individual subjects. Nowadays, diffusion-weighted MR tractography provides a more direct and noninvasive measure of this anatomical feature, and a number of researchers have incorporated tractography-based connection length values into their models (Izhikevich \& Edelman, 2008; Valdes-Sosa et al., 2009). Many still use non-tractography proxy methods, however, based on e.g. Euclidean distance (e.g.~Ghosh et al., 2008; Deco et al., 2009) or volumetric segmentation of T1-weighted MR images (Bojak et al., 2011). As a large proportion of this research is concerned with developing either forward (e.g. Bojak et al., 2011) or inverse (e.g. David et al. 2006; Valdes-Sosa et al. 2009) models that capture subject-specific anatomy and neural activity patterns, the relative accuracy of these various non-tractography or non-subject-specific proxy measures of fibre length is an important practical question. Similarly, a question that is certain to receive growing attention in the coming decade is how the `new breed' of MRI-based tissue microstructure metrics might contribute to this enterprise. Recent developments in structural MR imaging are making measurements of both axon diameters (Assaf et al., 2006, 2007, 2008; Alexander et al., 2010), myelin thickness (Stikov et al., 2011), myelin dystrophy (Avram et al., 2010). Whether these developments play a peripheral or a central role in future models of human brain dynamics will be a function of the precision of the measurements and the degree of precision required in the modelling process.


%$\subsection{Present studies}


%The aim of the present study was to evaluate systematically the accuracy and potential sources of variability involved in both current uses of tractography and potential future uses of microstructure imaging for anatomically-informed models of neural dynamics.

% We used linear uncertainty propagation techniques to assess the extent to which accurate knowledge of fibre length, diameter distribution, and myelin thickness contribute to accurate estimates of CDDs.  For this I drew on the results from a number of histological and neuroimaging sources. 

% Tract length measures are important for models of spatially distant interacting neuronal populations that incorporate axonal conduction delays. The reason for assessing the accuracy of the various fibre length approximations described in the previous section is that this directly affects the precision of conduction delay estimates based on these measures. However, track length is not the sole source of uncertainty in these delay estimates: myelin thickness, proportion of myelinated vs.~unmyelinated fibres, and the distributions of myelinated and unmyelinated diameters are all known with varying degrees of uncertainty, and varying levels of possibly for improved measurement accuracy. I

%n order to properly quantify the contribution of uncertainties in each of these quantities to the total uncertainty in
%conduction delay estimates, I conducted a series of formal uncertainty
%evaluations, in accordance with the rules and guidelines set out by the
%Joint Committee for Guides in Metrology (2008).
